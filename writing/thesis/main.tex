\documentclass[11pt, twoside, openright]{report} % 
\usepackage[utf8]{inputenc}
\usepackage{graphicx}
\usepackage[a4paper,width=150mm,top=25mm,bottom=25mm,bindingoffset=12mm]{geometry}
%\usepackage{biblatex}
\usepackage{setspace}
\usepackage[english]{babel} 
\usepackage{stmaryrd}
\usepackage{xcolor}
\usepackage{xspace}
%\usepackage{natbib}
\usepackage{csquotes}
\usepackage[style=apa, backend=biber]{biblatex}
%\DeclareLanguageMapping{american}{american-apa}
%\usepackage{apacite}
\addbibresource{references.bib}
\usepackage{tabularx}
%\usepackage{ltablex}

% packages for reading results
\usepackage{pgfplotstable}
\usepackage{csvsimple}
\usepackage{siunitx}
\usepackage{lscape}
\usepackage{amsmath}
\usepackage{amssymb}
\usepackage{mathtools}
\usepackage{hyperref}
\usepackage{adjustbox}
% define col width
\newcolumntype{Y}{>{\hsize=4\hsize}X}
\newcolumntype{s}{>{\hsize=0.25\hsize}X}
\graphicspath{ {images/} }

\linespread{1.25}
\counterwithout{footnote}{chapter}

\definecolor{Red}{RGB}{255,0,0}
\definecolor{Green}{RGB}{10,200,100}
\definecolor{Blue}{RGB}{10,100,200}
\definecolor{Orange}{RGB}{255,153,0}
\definecolor{Purple}{RGB}{139,0,139}

\newcommand{\denote}[1]{\mbox{ $[\![ #1 ]\!]$}}
\newcommand*\diff{\mathop{}\!\mathrm{d}}
\newcommand{\red}[1]{\textcolor{Red}{#1}}  
\newcommand{\eb}[1]{\textcolor{Blue}{[mht: #1]}}  
\newcommand{\mf}[1]{\textcolor{Orange}{[rl: #1]}}  
\newcommand{\pt}[1]{\textcolor{Purple}{[pt: #1]}} 

\DeclareMathOperator*{\E}{\mathbb{E}}
% define functions for reading results from csv
\newcommand{\datafoldername}{R4Tex}

% the following code defines the convenience functions
% as described in the main text below

% rlgetvalue returns whatever is the in cell of the CSV file
% be it string or number; it does not format anything
\newcommand{\rlgetvalue}[4]{\csvreader[filter strcmp={\mykey}{#3},
	late after line = {{,}\ }, late after last line = {{}}]
	{\datafoldername/#1}{#2=\mykey,#4=\myvalue}{\myvalue}}

% rlgetvariable is a shortcut for a specific CSV file (myvars.csv) in which
% individual variables that do not belong to a larger chunk can be stored
\newcommand{\rlgetvariable}[2]{\csvreader[]{\datafoldername/#1}{#2=\myvar}{\myvar}\xspace}

% rlnum format a decimal number
\newcommand{\rlnum}[2]{\num[output-decimal-marker={.},
	exponent-product = \cdot,
	round-mode=places,
	round-precision=#2,
	group-digits=false]{#1}}

\newcommand{\rlnumsci}[2]{\num[output-decimal-marker={.},
	scientific-notation = true,
	exponent-product = \cdot,
	round-mode=places,
	round-precision=#2,
	group-digits=false]{#1}}

\newcommand{\rlgetnum}[5]{\csvreader[filter strcmp={\mykey}{#3},
	late after line = {{,}\ }, late after last line = {{}}]
	{\datafoldername/#1}{#2=\mykey,#4=\myvalue}{\rlnum{\myvalue}{#5}}}

\newcommand{\rlgetnumsci}[5]{\csvreader[filter strcmp={\mykey}{#3},
	late after line = {{,}\ }, late after last line = {{}}]
	{\datafoldername/#1}{#2=\mykey,#4=\myvalue}{\rlnumsci{\myvalue}{#5}}}

% MH's command
\newcommand{\brmresults}[2]{\(\beta = \rlgetnum{#1}{Rowname}{#2}{Estimate}{3}\) (\rlgetnum{#1}{Rowname}{#2}{l.95..CI}{3}, \rlgetnum{#1}{Rowname}{#2}{u.95..CI}{3})}
%\brmresults{expt1_brm.csv}{condition}

\begin{document}
\begin{titlepage}
	\begin{center}
		\vspace*{1cm}
		\Huge
		\textbf{Language Drift of Multi-Agent Communication Systems in Reference Games\\} 
		\vspace{0.5cm}
		\Large
	
		\textbf{by \\ Polina Tsvilodub}
		
		\vspace{1cm}
		\small
		Submitted in partial fulfilment of the requirements for the degree of \\
		Master of Science in Cognitive Science \\ to the \\
		Institute of Cognitive Science at the Osnabrück University\\
		August 29th, 2022
		
	%	\vfill
		\vspace{2cm}
		Thesis Supervisor:\\ Prof. Dr. Elia Bruni, Institute of Cognitive Science, \\Osnabr\"uck University\\
		\vspace{0.5cm}
		Thesis Supervisor:\\ Prof. Dr. Michael Franke, Seminar f\"ur Sprachwissenschaften,\\ University of T\"ubingen \\  
		\vfill 
		\includegraphics[width=0.6\textwidth]{unilogo.jpg}
		
	\end{center}
\end{titlepage}

\chapter*{Abstract}
Teaching artificial agents to communicate about objects in the environment in a way natural to humans is difficult task. It has been approached in different domains, among which multi-agent communication research stands out in that there agents are trained to learn to communicate from interaction, while solving a task like a reference game. While previous works focused on properties of communication protocols emerging from scratch during task learning, or on investigating the effects of the communication protocol on the task success, this thesis fills the gap in hitherto considered task environments. More specifically, it aims to train agents to communicate about real-world images in a natural language (English), within a reference game \pt{which is a type of Lewis signaling game}. To this end, the MS COCO dataset is used for the main experiments \parencite{chen2015microsoft}. Following \cite{lazaridou2020multi}, the agents are trained using a learning signal consisting of both a task success-based loss and a language structure optimizing loss. Next to providing a proof of concept for the possibility to train agents on this task, this work focuses on investigating the reasons for the so-called \emph{language drift} phenomenon---the deterioration of the agents linguistic capabilities, as they are optimized for a task. Several possible reasons are investigated experimentally, along with investigating the sensibility of several new langage drift metrics. It is investigated whether varying the weighting between the two loss components affects the drift, whether the guaranteed presence of exhaustively descriptive captions is necessary for lower drift, \pt{whether the strength of the functional signal is important -- re vocab size, and if the decoding scheme and the speaker pretraining modes are important}. The results of the experiments indicate that...
For future work... 

\pt{references and potentially more definitions}

\chapter*{Acknowledgements}
%I want to thank...

I would like to thank...

\tableofcontents
%\listoffigures
%\listoftables

\chapter{Introduction}
\label{chapter01}
The way in which humans effortlessly communicate even in previously never observed situations is a fascinating phenomenon. Not only are humans able to communicate with each other about an infinite variety of contexts---they do so in a highly flexible and context-appropriate way. For instance, in a context where two persons are surrounded by many cars and houses in a lively street, intuitively, the speaker might utter a sentence like ``Look at that red polka-dotted car!'' in order to draw the listener's attention to a particular funny-looking car. In contrast, if the interlocutors were in a field where there are no cars but a single abandoned funny-looking one, the speaker might likely rather say ``Look at that car!'', ommitting details unnecessary for drawing the listener's attention to a particular object \parencite[cf.][]{graf2016animal, degen2020redundancy}. This \textit{context-dependent} selection of the necessary level of details in utterances is part of the basic communicative act of \textit{reference}, and presents only one example of how humans flexibly accomodate the requirements of a communicative task like reference in their use of natural language \parencite{searle1969speech, grice1975logic}.

Despite rapid progress in the area of artificial intelligence, teaching machines to communicate in a way understandable and natural for humans is still a far from solved task \parencite{lazaridou2020emergent, lake2017building, lecun2015deep}. Tackling this task involves at least three skills artificial agents need to learn: using natural language in a structurally well-formed way, using it in a semantically correct and grounded way, and being able to adjust their language use to the communicative task at hand \parencite{lazaridou2020emergent}. In context of multi-agent communication, \textit{grounding} is usually defined as aligning the natural language tokens agents use to visual input in the same way humans do (i.e., making sure that agents refer to images of cats with the word ``cat''; \cite{jurafsky2000speech}).

Modeling these skills has been addressed in different areas of research. For instance, work in computer vision focuses on the aspect of grounding and structural well-formedness in developing \textit{image captioning} models.
In contrast, training agents to communicate in order to complete tasks has been addressed in the growing body of work on \textit{multi-agent communication} in the reinforcement learning domain, where artificial agents are trained to develop or learn effective communication protocols \parencite[e.g.,][]{foerster2016learning, lazaridou2020emergent}.
This approach allows to both incorporate interactional aspects of language, as well as ground the language into the interaction environment. Furthermore, it allows to investigate the communication from an evolutionary perspective, as it evolves between the agents through interaction in context, allowing to investigate the roots of different communication protocol properties like compositionality \parencite{lazaridou2020emergent}. 

Multi-agent communication experiments have made use of different communicative environments. Some experiments investigate communication protocols emerging when agents have to play a game \parencite{jacob2021multitasking}, navigate in a 2D or 3D environment \parencite{das2019tarmac, jaques2019social}, learn to translate sentences \parencite{lee2019countering}, or let the agents play a variety of the Lewis signaling game---a \textit{reference game}. The goal hereby is for a sender agent to communicate in a way such that receiver agent(s) successfully identify a target object among distractors. The objects are typically represented by images. 

This thesis sets out to investigate \textbf{how artificial agents can be trained to use natural language in order to effectively refer to real-world situations in the reference game setting}. That is, presented work focuses on \textbf{training agents to produce \textit{discriminative} natural language and investigating the quality of the used language}. The reference game set up offers an ideal avenue for that by approximating the basic communicative act of referencing objects in the real world, thereby presenting a step towards teaching artificial agents skills necessary for successful communication in a human-like way. 

Previous work on multi-agent reference games has shown that artificial agents are able to develop communication protocols allowing them to successfully play reference games in a variety of different visual environments. More specifically, the agents have been trained to communicate about geometric shapes \parencite{ohmer2021and}, synthetically generated scenes \parencite{lazaridou2020multi} or even real photos of natural scenes in the MS COCO dataset \parencite{lazaridou2016multi, lin2014microsoft, havrylov2017emergence}. However, the communication protocols employed by the agents often are not human-interpretable; i.e., they communicate by emmiting symbols sampled from an arbitrary vocabulary \parencite{foerster2016learning, lazaridou2016multi}. Some experiments employed single word natural language labels of the objects agents communicated about \parencite{lazaridou2016multi}. Yet there are only few reference game experiments in which the agents use a full natural language (English) as their communication protocol \parencite[e. g.,][]{lazaridou2020multi}. However, \cite{lazaridou2020multi} who use full natural language made a compromise in complexity of the set up by applying the communication to simpler synthetically generated scenes from the \textit{Abstract Scenes} dataset \parencite{zitnick2013bringing}. This work aims to fill this gap by using both natural language and natural images in multi-agent communication experiments. 

Furthermore, it has been observed in the literature that training agents to complete certain tasks while sticking to a given communicative protocol does not come without difficulties. More precisely, when optimizing the agents' behavior by providing rewards based on task success, the agents \textit{drift away} from the communicative protocol, i.e., forget how to properly use natural language and produce unintelligible sentences like ``a a a red a cat.'' \parencite{lee2019countering, lazaridou2020multi, lu2020countering, lewis2017deal}. While some work has proposed ways to mitigate such \textit{language drift}, not much work has been done on investigating the precise reasons behind language drift, and especially not in connection with applying natural language communication to real-world visual input. 

To fill this gap in the literature, this thesis sets out to take a step towards \textbf{explaining language drift when employing natural language in reference games about realistic images}. The starting point for this thesis is the work by \cite{lazaridou2020multi}. The main  experiments focus on reference games on the MS COCO dataset, which provides \textit{real world images} annotated with English captions by humans \parencite{chen2015microsoft}, from which agents can learn to communicate about images in natural language. By using real world images, this work extends beyond previous studies in that it investigates language drift in a scenario which is close to potential interactions between artificial agents and humans in the real world. 
The dynamics of the quality of language used by the agents while learning the game are investigated with both extant and new language drift metrics. More specifically, new language drift metrics attempt to formalize the functional adequacy of the used language. 
%pairs experiment) of producing discriminative captions; failure to do taht could be dues to: noise in images, details of captions, incapability to vary caption length beyond what was observed in the training data
In order to investigate potential sources of language drift, the strength of structural contraints put on the agents' communication is varied in the experiments. Furthermore, the difficulty of the visual context is manipulated by varying the similarity of the target and distractor images in the game. This work also looks into the properties of the MS COCO dataset and whether the lack of discriminative details in the captions on which the model is trained is responsible for the models resorting to strategies which on the surface appear as language drift in order to complete the functional task. To this end, experiments on a manually annotated dataset which guarantees to include training captions of maximal descriptive granularity are conducted. The 3Dshapes dataset is used in these experiments \parencite{burgess20183d}. Using this dataset also allows to have a comparison to a dataset where the visual input is not as complex as MS COCO images because it depicts rather simple geometric shapes, as opposed to photographic images of many different scenes in diferent physical conditions. Finally, the speaker-listener agent co-adaptation as a potential source for language drift is investigated by conducting experiments wherein the speaker agent is trained against a fixed listener. This approximates an environment with a community of different listeners with which the speaker has to be able to communicate by maintaining certain linguistic conventions across individuals, as is arguably the case in human communication \parencite{kirby2014iterated, clark1991grounding}. 

Results from various experiments show that, while mostly being able to learn the reference game, the agents' task success was susceptible to the difficulty of the context, while their language quality was susceptible to the agents' co-adaptation, the speaker architecture details and the surface structure of the annotations in the training dataset. Overall, all considered drift sources entered into intricate interactions.

This thesis is structured as follows: first, some technical background is introduced in Chapter \ref{chapter02}. Then, related work on multi-agent communication is reviewed in Chapter \ref{chapter03}, especially focusing on experiments involving reference games. The reference game task is also discussed from a cognitive perspective, along with other modeling work on discriminative image caption generation. Chapter \ref{chapter04} then introduces language drift and reviews metrics employed to capture it, proposing novel approaches to detecting language drift in reference games on real-world images. Then, the experiments conducted in this work are presented in Chapter \ref{chapter05}. To this end, the architectures of the agents are explained, followed by a discussion of the employed datasets, training details, and results. Evaluation of language drift is presented alongside with the results. Finally, a general discussion of the results and their implications is provided in Chapter \ref{chapter06}. %The following sections presuppose faimiliarity with basic concepts in machine learning and deep learning. For an introduction, see, e.g., \cite{goodfellow2016deep, bishop2006pattern}.

%\pt{Have actual visual image and caption examples, and task conditional examples here.}

%\pt{maybe add basics of artificial agent interaction: \cite{tan1993multi}}

%\pt{a good general picture ML reference: Tomas Mikolov, Armand Joulin, and Marco Baroni. 2018. A roadmap towards machine intelligence.  Lecture Notes in Computer Science,}

%General flow: Correct, efficient communication appropriate in a given context as human feature -- basic type of cimmunication is reference (basic speech act), joint attention at a terget with another interlocutor. We want to model that, operationalized as a reference game which has also been studied extensively in cognitive science. We want artififical agents to do it, but for having potential to communicate with humans -- in natural language. It boils down to image captioning and learning statistical properties of language, yet in a task appropriate way. Therefore, multi-agent communication for task-specific finetuning. 

 \chapter{Technical Background}
 \label{chapter02}
 Literature review notes go here.

Make a table of definitions.

Sender / sepaker, receiver / listener, agents, message, distractor target. 

\pt{Summarise in detail base paper}
 
\chapter{Related Work}
\label{chapter03}

This chapter reviews related work on multi-agent communication---the core domain in which the presented experiments are situated. First, different architectures used for building the agent are reviewed. Then, different tasks the agents are trained on are summarized, focusing on the reference game setting. 
%Finally, the paper by \cite{lazaridou2020multi} which serves as the starting point for this thesis is summarized in detail.  

\pt{maybe find a way to say that there are other neighboring areas of research like visual dialogue, multimodal semantics, visual question answering etc. possibly in discussion.}

\section{Introduction}
Research on multi-agent communication has received increasing attention over the recent years, as this domain focuses on research questions of high relevance to cognitive science \parencite{lazaridou2020emergent}. More precisely, two main areas of research are often addressed: first, the conditions under which a language emerges and which emergent properties can be observed; and second, the ability of the agents to pick up a given language system. % thus investigating whether tractable human-agent communication channels can be learned via realistic interaction.
The second direction has also been considered with respect to \textit{grounding} a given language in the visual world. That is, approaches to teaching agents to use language applied to realistic visual input are investigated. Grounding is important both from modeling and cognitive perspectives: for instance, \cite{bruni2014multimodal} show that representations learned from both text and image data are more plausible than purely text-based representations; furthermore, vision is an integral part of human communication \parencite{tomasello2010origins, harnad1990symbol}. 

These two research directions have different implications for understanding human language: the former often focuses on what the causal factors behind prominent features of human language like compositionality are, allowing to shed light on human language evolution; the second rather addresses how we can train agents to communicate optimally for interacting with humans. Equivalently, these two main directions can be classified as investigating \textit{language emergence} and \textit{language acquisition} among and by artificial agents \parencite{lazaridou2018emergence, lazaridou2020emergent}.
Presented work focuses on the latter question, aiming to train agents to use English language grounded in real-world images.

Irrespectively of the research focus, artifical agents which interact with each other are at the core of multi-agent communication studies. Therefore, first, different agent architectures are presented in the next section.

\pt{I feel like a section on super general aspects, like having two agents doing some task and being jointly trained, is missing. Maybe visuals would be cool.}

Communication has been embedded in various tasks simulated in multi-agent settings. In some tasks like \pt{FILL ME} the idea is to use communication for more efficient cooperation and expertise transfer; in other tasks, communication itself is the core task. The latter allows to focus on properties of communicative systems that artificial agents might develop or acquire, thus allowing to compare them to human language. Since studying the ability of artificial agents to produce \textit{discriminative captions} for images in context within reference games is the goal of this work, this task and related work with human participants is described first, before transitioning to related work on multi-agent communication.

\pt{just to a section the chapter proceeds as follows and through out first paragraph and remove intro section.}

\section{Reference Games}
\label{reference_games}

Reference games are a type of the so-called \textit{Lewis singalling game} \parencite{lewis1969convention, skyrms2010signals}.
Signalling games were developed as an accont of \textit{conventional meaning} \pt{add def}. Traditionally, these games include two agents, a sender and a receiver. Only the sender observes a state sampled at random and chooses a message which is sent to the receiver. The set of possible messages is known to both agents. Based on the message, the receiver selects a state she chose based on the message. The game is a success if the guessed state and the state observed by the sender match, such that this game is driven by the agents having a common interest \parencite{lewis1969convention, franke2016evolution}. \pt{check if there might be a better reference}. 

\textit{Reference games} are then a specific instance of signalling games in which the meaning is already conventioanlly established. The sender samples a particular target among a set of distractors which may be a subset of all possible objects in the given world and \textit{refers} to it with her message. The target and the set of distractors are accessible to both agents. The receiver's task is then the identification of the target. Thus, the target and the distractors which are often represented by visual context in practice play a critical role in reference games.

Reference games have been employed in a wide range of studies with various research questions, often in order to study the pragmatic reasoning involved in generating referential expressions \pt{double check}. For example, \cite{franke2016reasoning} investigate whether pragmatic reasoning in reference games is population-level or individual. \pt{Def double check} For instance, \cite{graf2016animal} conduct an experiment wherein participants generate referential expression to a target image presented in context of distractors which either belonged to the same basic-level object category or to a different one.\footnote{\pt{define basic level categories and add reference, Rosch 1976}} They show that participants generated subordinate category, i.e., more specific, referential expressions in presence of more similar distractors, but not for dissimilar distractors. They show that humans flexibly increase the specificity of nominal expressions they use in order to unambiguously refer to the target, when required by the context. However, human speech also presents abundant examples of \textit{overmodification} in referential expressions, i.e., the use of additional modifiers like color expressions even when it is not strictly necessary for discrimination in the given conext. \cite{degen2020redundancy} explain this phenomenon in terms of complexity of the visual scenes, typicality of the referents and of the described features. They reveal complex pragmatic reasoning underlying this phenomenon. Much more work has been done on referential expressions both experimentally and theretically; these examples go to show that humans inherently generate highly discriminative, or, contrastive, messages in situations like reference games. Even further, they also expect and anticipate such behavior from other interlocutors, as has been shown in processing literature \parencite[e. g., cf.]{sedivy1999achieving}.

Applied to artificial agents, a reference game proceeds as follows: arrays containing a target image among $n \geq 1$ distractor images are sampled from the set of all images $I$. Both agents have access to the images and usually have the same representation; the sender agent knows which image is the target, the receiver does not. The sender emits a message $m, |m| \geq 1$ (assuming discrete communication; see Section \ref{multi_agent_arch}) sampled from the shared vocabulary $V$. Given $m$, the receiver has to correctly guess which of the images in the array is the target. Experiments in this thesis are limited to using one target image in context of only one distractor, but thy do not depend on this confinement.

\pt{Add Dale and Reiter, Computational interpretations of the Gricean maxims in the generation of referring expressions.  1995 }

\pt{Upshot: hypothesis about differing lengths as a proxy for different granularity of description. Bruni's paper as ground to believe that this might not be borne out as the bias in terms of length in the training dataset is too strong.}

\section{Discriminative Language Use}

Before turning to approaches wherein agents learn discriminative message generation in reference games, akin to human communication, rather machine learning-based approaches to building discriminative image captioning systems are reviewed.  \pt{it would probably make the most sense to review \cite{andreas2016reasoning} first, but then this framing doesn't quite make sense.}

\cite{sadovnik2012image} first approach discriminative image captioning by looking at visual discriminablity and saliency, and cosnstructing captions based on hand-crafted rules.

\cite{vedantam2017context} build an image captioning model which performs joint inference over a language model that is context-agnostic and a listener model which provides feedback on discriminativity of the captions. More precisely, they introduce a reaoning speaker consisting of a basic image captioning speaker and a listener model. The basic speakre is an image captioner pretrained on images and target concepts represented in the images. The listener is only dependent on the generative basic speaker, as it computes the log-likelihood ratio between the speaker's utterances for the same image but for different concepts. The overall speaker is the trained to maximize the basic generator probability while also maximizing the likelihood ratio represnting the discriminativity.
Thay apply the model to a justification task wherein the image feature corresponding to a given discriminative aspect of the caption has to be explained. They also apply it to standard image captioning on similar image pairs from the MS COCO dataset. They find that their architecture is able to generate caption which leads to higher discrimination success of human evaluators compatred to non-discriminative baseline.

\cite{dai2017contrastive} proposel the ``Contrastive Learning'' training objective for this task which constrains the image captioner to maximize the true caption likelihood compared to a pretrained reference image captioner, while learning to assign lower probabilities to mismatching caption for the target image, compared to that same reference model. Critically, the success of the system depends on the quality of the reference model. Their model achieves competitive results on the MS COCO test set.

While alleviating some issues like language drift \pt{check if it is okay to foreshadow this here like thsi}, these systems miss a critical component in the approach to the task---they are missing the communicative and interactive context of applying such cpations. Yet interactivity and social aspects of the communicative task are critical for developing systems which are supposed to be able to adequatly interact with humans. The arguably most prominent framework formalizing social and cooperative aspects of communication is the Rational Speech Act (RSA) framework \parencite{goodman2016pragmatic}. Some approaches to discriminative image captioning are highly inspired by RSA and thereby incorporate interactive communicative aspects into their models.\footnote{Basic knowledge of the RSA family of models is presupposed. For a gentle introduction, see, e.g., \cite{problang}.} \pt{double check adjectives here}

\cite{andreas2016reasoning} are the first to employ an RSA-style Bayesian model wherein the agents learn to produce discriminative messages in a reference game by reasoning about the other agent. They also use the Abstract Scenes dataset. The architecture consists of a base speaker, base listener and a reasoning speaker. All agent modules are based on fully-connected linear layers with non-linear activations. The literal listener consists of modules embedding input images and the received message, and computing scores over image-message pairs. The literal speaker is used to sample possible messages for a given image by computing scores over vocabulary tokens given the image with a feed-forward conditional language model. The reasoning speaker then derives the optimal message by sampling candidate messages from the literal speaker and reasoning about their utility by ``passing'' them through the literal listener and choosing the message maximizing the listener's referential success probability. 
\cite{andreas2016reasoning} found that referential success increases with the number of samples taken from the literal speaker, as well as with increasing weight of reasoning about the listener in the reasoning speaker model. Further, they find that the reasoning speaker outperforms the literal speaker, indicating that pragmatic reasoning might be crucial for modeling task-conditional language use. \pt{seems highly relevant to my intuition about regularization and strucutral weight. add to discussion.}

\cite{cohn2018pragmatically} also build a model following the standard RSA architecture. Their model performs character-wise caption generation in order to constrain the space of possile alternatives the speaker may choose among when generating the caption. Therefore, their generation process is an incremental one, modelled by an LSTM-based speaker agent. Their model  turns out to outperform word-level generation and non-pragmatic models. \pt{see if we want more detials}

\cite{nie2020pragmatic} propose an issue-sensitive image captioner, also based on the RSA famly of models. Issue-sensitivity is represented as partitioning  the space of images into cells with images having the same relevant (i.e., at-issue) feature as the target. Their base speaker is a standard  pretrained image captioning model. Additionally to standard RSA mechanics, they extend the pragmatic speaker to be sensitive to the relevant partitioning. They further include a utility term in order to tackle arising semantic drift (e. g., mentioning features that are not true of the target) of the captions and constrain the speaker from overgenerating. This model also employs an incremental speaker generation procedure. Evaluations include visual question answering (VQA) on MS COCO, showing that the model learns adequate issue representations from questions as well as generates promising issue-sensitive captions. 

Finally, an approach fully embracing the idea of learning language use from active interaction, not just representations of other agents,  is multi-agent communication. Therefore, the next section turns to work on discriminative language learning and other related research in this domain.

\section{Multi-Agent Communication}
\label{mac}
The main architectural differences concern the type of model representing the artificial agent and the type of communication channel they use.
As described in Section \ref{rl}, experiments wherein agents learn to complete a task by interacting with the environment and with each other are typically trained with reinforcement learning. Multi-agent communication is a specific type of such experiments in that (one of) the tasks agents learn are communicative and cooperative, and one of the agents can be considered the sender and the other the receiver of the communication \parencite[cf.][]{tan1993multi, lazaridou2016multi}.
Based on the chosen communication channel and the precise communicative task, specific architectural decisions and optimization algorithms differ. 

\subsection{Agent Architecture and Training in Reference Games}
\label{multi_agent_arch}

While early work in multi-agent communication focuses on structured agents \parencite[e. g., see][for reviews]{christiansen2003language, cangelosi2002symbol}, deep agent architectures became increasingly dominant in the domain over the last years \parencite{lazaridou2020emergent}. As this thesis also uses deep neural network agents, the review of related work focuses on deep multi-agent communication experiments. 

%Architectural considerations: 
%1) agent architecture: hand crafted vs deep; for deep: kind of network, esp. recurrent layer
%2) loss and training approach: Q-learning \parencite{foerster2016learning}, Gumbel-Softmax \parencite{havrylov2017emergence}, REINFORCE.  \parencite{lazaridou2020emergent, williams1992simple}. 

\cite{lazaridou2016multi} first show that neural agents can efficiently reference realistic visual inputs in the presence of distractors by developing a communication protocol from scratch. To this end, \textcite{lazaridou2016multi} conduct several experiments employing feed-forward and CNN agents. In particular, in their main experiment, they set up a reference game, where a sender agent emits a single discrete symbol referring to one of two images provided to both the sender and the receiver, sampled from the ImageNet dataset \parencite{deng2009imagenet}. The discretization of the agents' communication channel is sometimes also called the \textit{discrete bottelneck} \parencite{lazaridou2020multi}. 
The receiver agent has to guess  which image is the intended target, based on the symbol received from the sender (see Section \ref{reference_games} for details). %They use a subset of the dataset containing 100 images from each of 20 general categories.
\cite{lazaridou2016multi} set up two versions of the sender. The ``agnostic sender'' consists of a two-layer feed forward neural network, embedding both images and producing a score over the vocabulary given the concatenation of image embeddings \parencite[][p. 3]{lazaridou2016multi}. The ``informed sender'' consists of a three-layer neural network, applying two convolutional layers to embeddings of both images which are treated as different channels \parencite[][p. 3]{lazaridou2016multi}. It also outputs scores over the vocabulary. For both architecures the scores are converted into a Gibbs distribution from which the emitted message symbol is sampled. The receiver agent consists if a two-layer feed forward network which embeds both images and computes the dot products between these embeddings and the one-hot encoded message. The final target image choice is sampled from the Gibbs distribution computed from the dot products. %The agents are trained with REINFORCE which updates parameters of the speaker and listener policies by minimizing the negative expected reward. The speaker and listener policies are parametrised by the weights of the respective neural networks. The sender policy is $\pi(v \in V \mid i_L, i_R, t \in \{L, R\})$ where $V$ is the vocabulary, $i_L, i_R$ are the left and right images and $t$ is the position of the target. The receiver policy is $\pi(t \in \{L, R\}\mid v_s, i_L, i_R)$, where $v_s$ is the message received from the sender and $t$ is the guessed position of the target image. \textcite{lazaridou2016multi} set the reward to 1 if the image picked by the receiver is the target and 0 otherwise. 
They found that agents achieve high referential success, and that the informed sender produces semantically more natural symbols (i.e., the same ones for the same image categories) than the agnostic one. In a further experiment, the authors successfully pressure the agents to communicate about more high-level properties by sampling instances of different subcategories within the same basic-level category for the image pairs.%, and achieve a small increase in label purity (i. e., in the proportion of labels agreeing with the major cluster label). 
Finally, they perform an experiment grounding the communication in natural language wherein the sender agent has to use natural language category labels as message symbols. The sender is trained by switching between reference game play and image classification on ImageNet data. However, the vocabulary is limited to 100 words, and the sender only produces one-word messages. %They observe a higher symbol purity in this experiment, while retaining referential success. 	
%	\item They use REINFORCE for training the speaker. Overall framing: interactive communicative setting for training agents allows to capture functional aspects of communication, i.e., the goal-directed nature of communication, as opposed to purely supervised conversational agent training. 
%	\item \textit{multi-agent coordination communication game}: first functional task (a basic function of language): reference
	
Another foundational study in deep multi-agent communication was conducted by \cite{foerster2016learning}. In contrast to \cite{lazaridou2016multi}, the authors let the agents use a centrally trained system in their ``DIAL'' architecture (i. e., with parameter sharing across agents)  which can also be extended to continuous communication. This system is compared to the ``RIAL'' system where agents use traditional discrete communication \parencite[][p. 2]{foerster2016learning}. This is one of the first studies using a recurrent network in the sender architecture.
For both systems, the sender consists of an input network with a fully connected layer and lookup tables creating embeddings of the task environment, two recurrent GRU layers, and two feed-forward output layers.
The receiver has the same architecture. The DIAL agents are connected via a continuous vector which can be seen as an activation layer passing the internal state of the sender to the receiver.
Experiments with different variants of a riddle task and an image classification task on the MNIST dataset were conducted. Results on the former task show that the agents were able to learn an optimal policy although DIAL agents converged faster, yet sharing the weights between the sender and receiver agents was critical for RIAL. DIAL with parameter sharing outperformed other experiments. Furthermore, they found that adding noise to the communication was critical for successful learning for RIAL. Results of the second task provided similar results for DIAL, while RIAL failed to converge stably on the multi-step interaction version of this task. 
The discrete system is trained with deep $Q$-learning (see Section \ref{rl_methods}), while the continuous system was trained by additionally backpropagating gradients end-to-end between the two agents. %Furthermore, the former but not the latter case includes modeling wherein agents have access to each other's internal representations of the environment, which was also argued to be unnatural when developing a system to investigate human communication \pt{REF}. 
The DIAL results pose an interesting question regarding the comparability of such a system where the internal representations of a speaker are directly passed to the listener to human communication \parencite[cf.][]{lazaridou2020emergent, hockett1960origin}. The authors, however, argue that this direct error propagation can be interpreted as communicative feedback which is also part of human communication. Further, while the possibility to use DIAL with continuous communication is attractive from a machine learning perspective due to its straightforward differentiability, human languages are typically considered discrete signalling systems \parencite{hockett1960origin}. 

\cite{havrylov2017emergence} further extend work on multi-agent communication by modeling agents which  communicate with variable-length strings of symbols in a reference game. 
Their architecture is based on the proposal \cite{lazaridou2016multi} with the difference that the sender agent only has access to the target image. Furthermore, their agents consist of LSTM layers. Information about the target image is injected by initializing the hidden state of the sender LSTM with the image feature vector, extracted from a pretrained VGG CNN. The sender generates the message by sampling tokens from the vocabulary similarly to the image captioning system by \cite{vinyals2015show} described in Section \ref{image_captioning}. The vocabulary consists of 10,000 tokens, the LSTM hidden size is 512, and the token embeddings are 256-dimensional. 
The receiver interprets the message by encoding it with the LSTM and computing a probability distribution over dot products of image features and the affine-transformed hidden state of the LSTM. The image with the highest probability is chosen as the target. \cite{havrylov2017emergence} further attempt to make the statistical properties of the emergent protocol as similar to natural language as possible by minimizing the KL-divergence between the learned conditional token distribution and a pretrained language model.
They compare agents trained via the \textit{straight-through Gumbel-softmax estimator} to agents trained with REINFORCE. The straight-through Gumbel-softmax estimator is a differentiable approximation of discrete actions (i. e., tokens) with a continuous relaxation applied in the backward pass of the backpropagation, while still using discrete representations in the forward pass. 
That is, one-hot encoded tokens are replaced with samples $w_k$ obtained by sampling $K$ samples  $\{u_k\}_{k=1}^K$ from the random variable $u \sim U(0,1)$. Each $u_k$ is transformed and then used in the sample computation:
\begin{equation}
\begin{aligned}
g_k = -log(-log(u_k)) \\
w_k = \frac{exp((log \; p_k + g_k) / \tau)}{\sum_{i=1}^{K} exp((log \; p_i + g_i) / \tau)}
\end{aligned}
\end{equation}
where $\tau$ is the temperature, $p_i$ and $p_k$ are token probabilites. The samples are discretized with argmax for the forward pass. Thus, it is an alternative to REINFORCE in this setting.
It is argued in the literature that this approximation allows more efficient gradient estimation given larger action spaces for which REINFORCE produces very high variance estimates, e. g., when the vocabulary size is larger \parencite{havrylov2017emergence}.

Experiments on the MS COCO dataset are conducted, selecting one image in batches of 128 as the target. They showed that agents trained with Gumbel-softmax converge faster than those trained with REINFORCE, but both are able to achieve high referential success. Further, they find that communication success increases with maximum message length for both algorithms. They also find that the emergent communication protocol has multiple representations of the same information and that the messages seem to have hierarchical coding. They also conduct an experiment wherein the sender is trained with a combined loss on both an image captioning and reference game task in order to attempt to grounding the tokens in the images. However, this system didn't improve in terms of caption quality. To sum up, this work showed that agents can successfully develop multi-token communication protocols for solving referential tasks.

Similar work is conducted by \cite{lazaridou2018emergence}. The agent architecture closely matches \cite{havrylov2017emergence}, yet the study focuses on comparing reference games with realistic images to games on symbolic data. The protocol learned on symbolic data was shown to exhibit higher topographic similarity to the input data and more compositional features compared to raw pixel input experiments. Nevertheless, agents were able to successfully develop a protocol based on raw input, as well. These results suggest that the proposed architecture does not have an inductive bias sufficient for extracting symbolic compositional features from raw inputs. 

\cite{lazaridou2020multi} whose work is replicated and extended in this thesis use an architecture which combines the architectures by \cite{lazaridou2016multi} and \cite{havrylov2017emergence} in that they also use LSTM based agents, but condition them on combined image features of both the target and the distractor. That is, they conduct reference game experiments with pairs of images. They explore different speaker architectures. More precisely, they compare a speaker learning an emergent communication protocol maximizing the reference task success (``functional learning''), a speaker learning to emit images captions as messages based on ground truth captions in a supervised manner (``structural learning''), and speakers trained with a combination of the two learning approaches \parencite[][p. 4]{lazaridou2020multi}. The functional learning signal is also based on REINFORCE, while the structural learning is conducted with the standard cross-entropy loss. Multiple combined speaker parametrizations are compared: a speaker pretrained on an image captioning task which is then fine-tuned with functional learning on the reference game (conditioned on the target image only) (``reward finetuning''), a speaker trained with a weighted combination of the structural and functional losses (``multi-task learning'') and a speaker pretrained on image captioning and then learning a reranking function based on the reference game (``reward-learned rerankers'') \parencite[][p. 4--5]{lazaridou2020multi}. Two variations of the reranking model are presented; both speakers learn to rerank samples obtained from a pretrained image captioner. The ``product of experts reranker'' reranks the samples proportionally to the message probability times the probability of the message given the image pair. The latter is obtained by re-embedding the sampled captions with a trainable layer as bag-of-words vectors and computing dot ptoducts between the vectors and image embeddings, renormalizing the scores.  The ``noisy channel reranker'' learns the reranking according to Bayes rule, i. e., proportionally to the likelihood of obtaining the target image given the sample times the sample probability. The likelihood can be seen as the speaker's internal listener model and is also computed as the re-normalized dot product between a learned bag-of-words caption representation and each image embedding \parencite[][p. 5--6]{lazaridou2020multi}. The agents were trained on a reference game on the \textit{Abstract Scenes} dataset.
They showed that the ``product of experts reranker'' model outperforms other speakers with respect to referential success in the game, while also counteracting language drift (see Chaper \ref{chapter04} for details). 

More specifically, they conduct a set of evaluations: they evaluate the different trained speakers trained simulataneously with a listener against the same joint listener, a fixed pretrained listener and a human, using an easy and difficult test set consisting of 1000 image pairs each. Furthermore, they vary the source of the samples for the reranker models: they are either the ground truth captions or samples from an image captioner pretrained on single image captioning only. As a baseline, they have human speakers pick the most discriminative caption among the ground truth options for a human listener and achieve an acuracy of 0.97. By contrast, an oracle reranker model operating on the ground truth captions learned against a joint listener only achieves an accuracy of 0.92 with human listeners, indicating that the capacity of the listener and reranker modules are somewhat below human performance. They find that the noisy channel speaker performs best ith humans when trained with a joint listener, while the PoE model performed best with humans when reranking ground truth captions. Investigating the reasons for language drift, they find that the speaker-listener co-adaptation has the least effect on the noisy channel speaker (human performance 0.86 vs 0.87 for joint vs fixed speaker training), and most significant effect on the the reward finetuning with KL speaker (human performance was 0.69 vs. 0.0.75). Finally, they find dthat unfreezing the visual model weights increses the gap between the performance of the joint and human listeners when training the reranker with a joint listener on with ground truth captions. 

Experiments in this thesis focuses on the ``multi-task learning'' architecture. This choice is discussed in more details in Chapter \ref{chapter05}.
\pt{make sure to describe in detail the results of easy-difficult comparison and tests against oracle speakers.}

%The noisy channel reranker model by \cite{lazaridou2020multi} is closely related to work by \cite{andreas2016reasoning}. They also employ a Bayesian model wherein the agents learn to produce discriminative, i. e., task-directed grounded messages in a reference game by reasoning about the other agent. They also use the Abstract Scenes dataset. The architecture consists of a base speaker, base listener and a reasoning speaker. All agent modules are based on fully-connected linear layers with non-linear activations. The literal listener consists of modules embedding input images and the received message, and computing scores over image-message pairs. The literal speaker is used to sample possible messages for a given image by computing scores over vocabulary tokens given the image with a feed-forward conditional language model. The reasoning speaker then derives the optimal message by sampling candidate messages from the literal speaker and reasoning about their utility by ``passing'' them through the literal listener and choosing the message maximizing the listener's referential success probability. This architecture is closely related to idea of Rational Speech Act models \parencite{goodman2016pragmatic}. 
%\cite{andreas2016reasoning} found that referential success increases with the number of samples taken from the literal speaker, as well as with increasing weight of reasoning about the listener in the reasoning speaker model. Further, they find that the reasoning speaker outperforms the literal speaker, indicating that pragmatic reasoning might be crucial for modeling task-conditional language use. \pt{seems highly relevant to my intuition about regularization and strucutral weight. add to discussion.}

\cite{lee2019countering} use another architecture---an attention based sequence-to-sequence machine translation model with one GRU layer---for both agents. 
The agents play a multi-modal translation game wherein the first agent is tasked with the translation of a sequence from French to English and the second---from English to German. Both agents are pretrained on the translation task before the communication game. This setup is used in order to investigate \textit{language drift} of the pivot language English, i. e., its deterioration (see Chapter \ref{chapter04} for details), as the agents are trained on French-German translation with REINFORCE (see Chapter \ref{chapter02} for details). They compare the mitigation of language drift through a language modeling constraint (i. e., incorporating the maximization of the likelihood of the English sentence under a pretrained LM into the reward) to grounding (i. e., maximizing the likelihood of an image corresponding to the English sentence under a pretrained image retrieval model). The LM is a one-layer LSTM; in the grounding model, image features are extracted using a pretrained ResNet-152. \cite{lee2019countering} find that the agents are best able to learn decent French-German translation when mitigating language drift with both an LM constraint and grounding. \pt{check italics of language drift and reference to chapter 4 thru chapter}

\subsection{Other Tasks}

Next to studying communicative success and the ability of agents to use natural language in reference games, multi-agent communication work has also focused on other aspects in reference games as well as studied other tasks.
 
For instance, \cite{evtimova2017emergent} extend the one-iteration reference games to \textit{multi-step interactions}, wherein agents can exchange information back and forth several times. More specifically, their sender agent only has access to visual representations of the set of objects, while the receiver only has access to textual descriptions of the target and distractors. Therefore, the agents need to align their communication across modalities. They compare a feed-forward and an attention-based sender architecture, and a GRU-based and attenion-based receiver, training them with REINFORCE. They find that agents successfully make use of back-and-forth communication about ImageNet images with WordNet based descriptions, showing that agents produce longer conversations on more difficult concepts. Further, they observe an increase in sender's message specificity with progressing conversation iterations.
Their experiments are also related to the multi-step MNIST task variation of \cite{foerster2016learning}.

A further step is taken by \cite{bouchacourt2019miss} who model a decision task wherein one agent is assigned a fruit and the other two tools, the latter having to decide which tool is more suitable for using with the given fruit. The tool utility is retrieved from human judgements. Noteworthily, the type of objects assigned to an agents is varied at random. They observe that the agents develop role-dependent communicative protocols. 

Other work studies the communication in more complex cooperative environments like 2D or 3D grid worlds \parencite{das2019tarmac}. In contrast to other studies, they use the Actor-Critic training method and continuous communication protocols. In short, they find that the benifit of including communication increases with task complexity, and that the learned communication and underlying agent behaviour are intuitive and interpretable. However, \cite{lowe2019pitfalls} also highlight the difficulty of adequately evaluating the added value of communication in complex multi-agent environments.

Finally, a body of work focuses on investigating the properties of emergent languages with the goal of understanding how natural language properties might have developed \parencite{lazaridou2020emergent}. For instance, \cite{graesser2019emergent} show, among other findings, that a shared communication protocol emerges from distinct ones in presence of a single new agent in the community participating in the communication. This presents an important connection to natural language which largely preserves its structure precisely because of the pressure to communicate with different interlocutors and culturally transmit the language to further generations. \pt{reference! and check!} \cite{chaabouni2019anti} investigate whether emergent communicative protocols follow Zipf's law positing that freuquency of messages in a language is anti-correlated with message length. Surprisingly, they find that emergent protocols are \textit{anti-efficient}---that is, the messages that have to be used most frequently are the longest ones. This is explained by a lack of production cost pressure in emergent communication.

Similarly, driven by investigating properties of natural language, a lot of work has focused on \textit{compositionality}.\footnote{Measuring compositionality is an important and not trivial aspect of this line of work. For an overview of different approaches, see, e.g., \cite{lazaridou2020emergent}.} For instance, \cite{lazaridou2018emergence} show that compositional communication emerges more easily from symbolic input than raw pixel input. 
\cite{chaabouni2020compositionality} show that compositionality in emergent protocols facilitates language transmission, but is not predictive of its generalization potential. The emergence of compositionality is shown to be strongly dictated by the variability of the input environment. Importantly, they also use a structured symbolic input representation.
This is in line with the observation made in human experiments and grounded in theoretical work that ``generational transmission of language favors compositionality'' (\cite{lazaridou2020emergent}, p. 12; cf. \cite{kirby2014iterated}).
\cite{luna2020internal} investigate cognitively plausible internal and external pressures which might influence compositionality---the ``principle of least effort'' (i.e., keeping the messages as short as possible) and ``object constancy'' (i.e., grouping together constant patterns into conceptual classes by abstracting away from context-contingent variation) (p. 1). More precisely, operationalizations of the latter principle in terms of sensitivity towards location invariance, color constancy and the distribution of objects in the environment are compared. The results show that the least effort pressure makes the protocols less redundant. Furthermore, agents pressured towards object constancy produce protocols with highest compositionality scores, while some operationalizations also improve zero-shot generalization abilities. 

To sum up, these studies show that emergent communicative protocols often exhibit properties that are far from natural language. Nevertheless, the ability to test the effect of different environmental and architectural pressures on emergent properties makes multi-agent communication a fascinating avenue for further research. 
%Other evolutionary perspective: agent co-evolution \parencite{dagan2020co}.

%\textbf{Properties of the communication \parencite{lazaridou2020multi}}: difficulty to converge on consistent meanings of tokens for color values, generally task-oriented codes with complexity minimization. Anti-efficient code development --> I am looking somewhat into this with my different-lengths experiments and grammars.  Larger communities lead to more systematic languages --> fixed listener experiment.

\pt{it would be good to have something with different listeners, i e the cultural transmission aspect.}
%Compositionality. also paper by Bruni.

\pt{Don't forget chapter summary.}

\chapter{Investigating Language Drift}
\label{chapter04}
This chapter introduces the notion of language drift and reviews work investigating this phenomenon. Literature often distinguishes between \textit{syntactic} (also sometimes called \textit{structural}), \textit{semantic} and \textit{pragmatic} drift \parencite{lazaridou2020multi}. \pt{add some relation or examples from human language development, cf. \cite{jacob2021multitasking} p. 3.}

First, metrics aiming to capture these different kinds of drift which are also used in presented experiments are reviewed. Then, novel aspects in investigating language drift tested in this thesis are described. Finally, specific hypotheses regarding expected language drift in the presently conducted experiments are presented along with their operationalizations. 

\section{Mitigating Language Drift}

\pt{maybe add images with examples of diverse drift}
The phenomenon of \textit{language drift} was first detected by \cite{lewis2017deal} who train artificial agents to cooperate on a negotation task in natural language, given a corpus of human dialogues. They show that the negotiation skills are significantly improved by optimizing the pretrained sequence-to-sequence agents with a preformance-based reward via REINFORCE. However, this comes at a cost of divergence from human language---i.e., the human intelligibility of the communication produced by the agents drastically decreases. This phenomenon is called language drift. To combat that, they switch between reinforcment learning and supervised learning. However, no precise quantification of the drift is presented. 

Building upon this work, \cite{lee2019countering} observe that counteracting drift by imposing a supervised learning constraint on the produced language, i.e., by trying to maximize the likelihood of the communication under a pretrained language model, mainly the \textit{surface structure} of the learned language is preserved. However, there is no guarantee that the \textit{semantics} are not drifting. That is, there is no constraint for the model to mean a cat with the word ``cat''; it could instead attach a different meaning to it.  To address this issue, they suggest to ground the language usage task in visual data, the idea being that the by co-occuring with images, the words keep the semantics. As described in Section \ref{mac}, they cast a translation task into a multi-agent task wherein the agents communicate via the pivot language English which is used for investigating drift. It is also used to compare the efficacy of grounding to a direct lanuage-model likelihood regularization, as performed by \cite{lewis2017deal}. \pt{check if regularizer description should be ported over here.}. Interestingly, the visual model they use for grounding is an image-caption retrieval model, determining the likelihood of the target image under the produced message, in  contrast to the grounding model by \cite{lazaridou2020multi} where the likelihood of the message given the image is maximized.
They first estimate language drift within the evaluation of the overall translation quality by reporting BLEU scores. The observe a drop in French-Englis translations in the vanilla model when the scores for the French-German performance increase. The highest scores were achieved with the model containing both constraints both on French-Eglish translation and the French-German translation. LM regularization has shown better improvement of the English scores, though this is partly due to the preference of the BLEU score for the surface form which is better regulated by an LM. Yet combining the LM and the image retrieval model yielded the best results, indicating that the drift may also occur at semantic level which is mitigated by grounding. Second, they look at by part-of-speech recall at inference time, finding that the vanilla model has difficulties with function words as well as produces in a flat toke distribution, compared to the combined regularized model. Further, the vanilla model is more prone to repeating words. To sum up, their results hint that both syntactic and semantic drift take place when optimizing agents with external task reward. However, they capture it by task-specific metrics (i.e., translation BLEU scores) which might make it difficult to extend their diagnostics to other experiments.

A different approach to mitigating language drift, focusing rather on stabilizing the language by creating community pressures in the agents is taken by \cite{lu2020countering}. They test the so-called \textit{seeded iterated learning} approach whereby they periodically finetune the pretrained speaker agent to imitate the behaviour of a teacher agent finetuned on the task. It is motivated by iterated learning models which are prominent in research on emergence and evolution of language structure. More precisely, the teacher agent is initially a duplicate of the learning speaker agent which essentially provides the \textit{seed} for the supervision iterations. The imitation sequences consists of supervised training on data sampled from the fnetuned teacher agent. They test this approach on a reference game and the translation game from \cite{lee2019countering}. The success of countering the drift in the reference game is measured via the ``Sender Language Score'' and the ``Task Scores'' (p. 5). The former compares the generated ang ground-truth captions by-token. The latter is the referential accuracy. The SIL model is shown to outperfrom several baselines.
For the latter task it is measured via BLEU scores, negative log likelihood of the messages (NLL, for capturing structural drift) and ranking scores under a pretrained image ranker (for capturing semantic drift). The SIL model is shown to be more robust against BLEU and ranking and NLL decrease than baselines.
%--  evolutionary / community / stabilisation via listener variation / against coadaptation.

Last but not least, \cite{jacob2021multitasking} study semantic drift of the so-called \textit{latent language policies} (LLP) which are used to train instructor and executor agent pairs. Semantic drift refers to the phenomenon whereby instuctors use messages in ways inconsistent with their initial semantics. Applied to the signalling game setting, they propose an executor (i.e., speaker) which simulataneously receives reward functions based on two different tasks and two different executors (i.e., listeners), respectively. Yet it remains a task for future research if this approach can be applied to training agents in the reference game setting.

\pt{from proposal below.}
The approaches outlined above test specific architectural constraints tailored towards reducing language drift. The goal of this thesis, however, is to take one step back and first effectively \textit{measure} language drift within a given architecture in order to \textit{explain} its possible sources, before making some recommendations about potential architecture improvements. Therefore, the next section summarizes existing and newly applied language drift metrics which are employed in own experiments.  

\section{Measuring Language Drift}

Some language drift metrics have been identified in the reviewed work. In particular, \cite{lazaridou2020multi} identify several measurements which are adopted in this thesis. These can be summarized as follows. 
\begin{itemize}
	\item \textit{Structural}, or, syntactic language drift can be measured as the log probability $P(m)$ of the generated message $m$ under a pre-trained unconditional language model. In this thesis, the pretrained TransformerXL model accessed through the \texttt{huggingface} library is used \parencite{dai2019transformer, wolf2019huggingface}.
	\item \textit{Semantic} language drift can be measured as the conditional log probability $P(m|i)$ of the generated message $m$ given the image $i$. %Another measure includes the $n$-gram overlap of generated messages and the ground-truth captions (ignoring stopwords) \cite{lazaridou2020multi}. Semantic drift is also addressed by \cite{lee2019countering}, \parencite{} but their approaches rather propose specific training methods than measures for identifying language drift, so their proposals wouldn't be considered here.
%	In alternative framings, semantic drift has been measured as the difference between the message semantics and the action taken by the receiver agent \cite{}. lu2020countering.
	\item Finally, \textit{pragmatic} language drift is assessed as the discrepancy in referential success in absense of structural or semantic drift between human and listener agents. This is assessed by comparing the performance of humans to the performance of trained listener agents with a speaker trained to rerank the ground truth image captions (see Chapter \ref{chapter03} for details). 
\end{itemize}

However, given that the conducted experiments won't have access to human baseline data, pragamtic drift will have to be assessed differently. Although the proposed approach has the advantage of being task-agnostic, this work proposes to focus on a referential task drift approximating pragmatic one. This approximation is referred to as \textit{functional} drift. 

In this context, funtional drift refers to the deterioration of language which would make the referential task impossible for humans (e.g., leaving out critical content words). Crucially, the the goal is to capture this drift in absense of human experiments. The difference between the proposed metric and pragmatic drift is that functional drift is proposed in terms of the presense of discriminative words in the discription, which can be approximated as words or caption parts which have a higher probability for the target image than for the distractor. Thereby the metric becomes operationalizable under open-source pretrained models and does not depend on the availability of human data anymore. \pt{check if all this is correct, makes sense, aligns with the actual metrics I use and whether it is different from pragmatic drift.}
In contrast, other kinds of drift like structural drift might involve mixing up the word order, which nevertheless does not necessarily hinder the referential task, if distinctive content words are still present. \pt{For instance, the caption ``A plate food with'' would exemplify functional drift, while the caption ``A plate red food with'' wouldn't, if was the target image and Fig.2 was the distractor. Check is this can be exemplified with some images already used elsewhere or at intro of the chapter.} 

The following concrete operationalizations are tested in the experiments in order to capture functional drift. \begin{itemize}
	\item One idea approach for identifying functional language drift which is stable against compositional alternations within the caption and, therefore, isolates functional discriminativity is the word overlap between the generated captions and the target and distractor ground truth captions, respectively. From a functional perspective, an optimal generated target caption maximizes the overlap with the target ground truth, while minimizing the overlap with the distractor ground truth. This idea is related to the omission score suggested by \cite{havrylov2017emergence} (also cf. \cite{andreas2016reasoning, gunel2020supervised}). This is also similar to the unigram metric employed by \cite{lazaridou2020multi}, but it adds the functional aspect via the difference computation. To sum up, the difference between the word overlap of the target and generated captions and the overlap of the distractor and generated captions is computed. This is an intuitive step to take since ground truth captions are available for all images in the dataset.
	\item Complementarily, the idea described above is also formalized by computing the cosine similarity between the caption embeddings instead of word overlap scores. This also hints at whether the trained embedding layer of the speaker, i.e., a representational layer, is affected by the functional learning signal. Again, the respective difference is computed as the drift metric and is expected to increase with successful task learning.
	\item Finally, a rather exploratory approach is taken to measuring the recoverability of the target image based on the caption. \pt{add an explanation of recoverability} Again, discriminative captions would present higher recoverability of the target comapred to the distractor. \pt{This is operationalized via a pretrained image-text retrieval model. Alternatively, this can be operationalized via a text-to-image model, where, ideally, the image produced from the generated caption would be more similar to the target than the distractor image. However, these metrics both heavily depend on the nature of the pretrained models as well as in the chosen similarity metric. This metric is rather used in an exploratory way, in order to investigate if it alignes with intuitions and other metrics.}
	\item \pt{check if i do some POS based tests}
\end{itemize}
These metrics are used as diagnostic measures in order to identify potential sources of language drift. To this end, different experiments are conduceted. The section below outlines the hypothesized language drift behavior in the different experiments, as can be expected based on results from the literature.

\section{Hypotheses}

For simplicity of reference in the discussion of results, the different hypotheses are enumerated.
First, hypotheses regarding language drift in the main experiments on the MS COCO dataset are described. As the speaker improves on the functional task, i.e., as the functional loss is minimized, the following observations are hypothesized: \\
\newline 
% The speaker vocab size relation also to previous work 
\textbf{H1:} The structural drift is expected to increase. That is, the log probability of the generated captions under a pretrained model is expected to decrease. \newline
\textbf{H2:} The semantic drift is expected to increase. That is, the log probability of the generated caption given the target under the pretrained frozen speaker model is expected to decrease. \newline
\textbf{H3:} The discriminativity of the captions is expected to increase. That is, the word overlap difference is expected to increase, both for the word overlap based metric and the cosine similarity based metric. \newline
\textbf{H4:} Given that the pressure to stay close to a natural language distribution is provided by the structural loss component in the chosen model architecture, it is expected that both the structural and the semantic drifts increase as the weight of the structural loss component decreases. To investigate this, four reference games with the strctural loss weights $\lambda_s = [0, 0.25, 0.5, 0.75, 1]$ are conducted. the functional loss component is computed as $\lambda_f = 1 - \lambda_s$. \newline
%Based on exploratory experiments (see Appendix \ref{appendix}) and on literature regarding the learnability of large action spaces by REINFORCE, 
\textbf{H5:} The discriminativity of the captions is expected to increase less than in the baseline experiment on random target-distractor pairs. That is, the overlap values are expected to be smaller than in the baseline experiment. This is expected due to the higher perceptual difficulty of discriminating images depicting similar things. Due to the intuitive necessity to produce more specific messages in the similar pairs experiment, it is also expected that the message length and possibly the specificity of the words increases. This will be approximated by analysing the occuring parts of speech and message lengths until the first END token, possibly accompanied by manual sample message inspection.\newline
\textbf{H6:}  The semantic and structural drifts are expected to be smaller in experiments where in the speaker is trained against a fixed (i.e., pretrained) listener compared to a listener trained jointly with the speaker. This is due to the observation made in prior work that especially semantic drift arises due to speaker-listener co-adaptation and the emergence of conventions among them. \newline
\textbf{H7*\footnote{The * indicated the optionality of this hypothesis based on available time}:} \pt{MS COCO vocab size, if there is time. What exactly is the language drift H here though? It is more about the functional optimization potential} \newline
\textbf{H8*:} The recoverability of the target given the sampled caption is expected to increase. That is, the metric provided by the image-text retrieval model is expected to improve. \pt{add direction and check} \newline

\pt{maybe emphasize once again that this cannot plausibly be expected if no functional learning is observed, re reinforce thing.}
In order to investigate the influence of the characteristics of the dataset on the drift, experiments with the manually annotated 3dshapes dataset are conducted. Hypotheses ]textbf{H1--H6} are also tested on 3dshapes, but additional hypotheses about the comparison between MS COCO and 3dshapes are explored. \\
\newline
\textbf{H9:} The strucutral drift is expected to be smaller compared to the MS COCO experiments. That is, the message log probability under the pretrained language model is expected to be higher. This is expected due to the a priori given availability of exhasutively descriptive features in the training data. This availability is expected to alleviate the need of resorting to structurally misformed messages in order to adapt to the functional needs for the agents. \newline
%Message length, grammatical structure. 
%Granularity in random vs similar pairs.
%Again Ls, difference between joint and fixed listener.
\textbf{H10:} To deconfound the role of the availability of fully exhaustive descriptive captions and the size of the action space learned by the agent, an experiment is conducted wherein the speaker is pretrained on non-exhaustive captions. Comparing this experiment to the baseline 3dshapes experiment, higher structural drift is expected, as operationalized by lower cpation log likelihoods of the messages under the pretrained models.

It is important to note that absolute values of the presented drift metrics should be interpreted with caution. Especially the structural drift metric is based on a model pretrained on possibly very different data distributions. Therefore, absolute log likelihood might be artifacts due to differences between the reference games data and pretraining data. Therefore, the metrics are intended to be interpreted comparatively within the training dynamics of the experiments, or between experiments varying training configurations.

The next chapter finally turns to experiments in scope of which language drift is investigated. \pt{maybe chpater summary}

\chapter{Experiments}
\label{chapter05}
This chapter first describes the two datasets MS COCO and 3Dshapes used in this work. Then, the agents' architecture is presented. There are two agents---the speaker and the listener agents---whose architectures are shared across all experiments. Finally, the single experiments and their results are discussed.

\section{Datasets}

This work uses two datasets for the reference game. The dataset chosen for the main experiments is the MS COCO Captions dataset \parencite{chen2015microsoft}. As it is one of the most widely used image datasets containing real photos of common objects and scenes, this choice was motivated by the goal of this work is to test multi-agent communication on natural images annotated with natural language captions. 

Further, a second dataset was chosen for conducting baseline experiments in order to investigate potential sources of language drift---the 3Dshapes dataset developed by \textcite{burgess20183d}. This dataset was chosen due to its systematic content and exhaustive annotations of all relevant image features, allowing to automatically generate exhaustive natural language captions for the dataset. This dataset has by far less features and categories varying across the images compared to the main dataset, providing a baseline for estimating the models' performance in a more controlled environment.

Both datasets are used for constructing the reference game which procedes in the following way:
\begin{enumerate}
	\item A random image is sampled from the dataset as the target $t$.
	\item A second image is sampled as the distractor $d$ (either at random or following constraints; see below for details).
	\item The speaker agent $S$ recieves both images, the target image being marked, respectively. This is accomplished by always passing the target as the first image. It emits a message $m$ describing the target.
	\item The listener agent $L$ recieves both images along with the message $m$, but doesn't know which image is the target.
	\item The listener selects the target image based on the message. If the selection was correct, both agents receive a positive reward, and a negative one otherwise.	 
\end{enumerate}

The next two sections describe in detail the properties of the datasets and how they were preprocessed for the experiments.

\subsection{MS COCO}
\label{ds:coco}
The MS COCO Captions dataset \parencite{chen2015microsoft} contains images and respective annotations from the 2014 split of the MS COCO dataset \parencite{lin2014microsoft}. Figure \ref{fig:coco_example} presents an example image and annotations form the dataset. The dataset contains 82,783 images in the training split, each associated with around five human-annoated caption, resulting in a total of 414,113 unique image-caption pairs. The validation split contains 40,504 images, annotated with a total of 202,654 human-produced captions. The respective dataset splits and annotations were downloaded from \url{https://cocodataset.org/#download}.
Furthermore, each image is annotated with bounding boxes of objects and category labels for the depicted objects. There are 80 different basic-level annotation categories, listed in the Appendix \ref{appendix}. These 80 categories are grouped into 12 superordinate categories: person, accessory, indoor, appliance, food, kitchen, sports, furniture, outdoor, animal, electronic and vehicle.
On average, 2.9 basic-level categories and 2.3 superordinate categories occur per image.
\pt{There will be a table with super-basic category mappings and counts of at least each super category}.

The natural language captions associated with the images contain a total of 24,697 unique tokens. Yet over 99\% of the words occuring in the entire dataset are covered by 6000 most frequent tokens. 
The final size of the vocabulary $V$ used in the experiments is 4054, including four special tokens \texttt{START, END, UNK, PAD}, as it comprises over 98\% of the token distribution mass, while comprising 16.4\% of unique tokens occurring in all captions. This vocabulary size was chosen as a good trade-off between a sufficient variety of words that the speaker can choose to describe the images and an action space size which is still learnabale in the current set-up. 

The minimal caption length occuring in the dataset is six tokens, the maximal length is 57 tokens. The mean caption length is 11.3 tokens (based on tokenization with the basic English tokenizer provided by the \texttt{torchtext} package). Since recurrent neural networks may have issues with learning very long sequences \pt{REF}, captions exceeding the length of 15 tokens were truncated. This cut-off length was chosen because it is the minimal length at which over 99\% of the captions didn't have to be truncated. %\pt{maybe include a density plot, side by side with the vocab count density one}
During preprocessing, the captions were lowercased, tokenized with the \texttt{torchtext} tokenizer mentioned above and mapped to numerical indices.  The resulting vocabulary is shared between the speaker and listener agents in all experiments. 

The following preprocessing steps were applied to all images before training: the images were resized to 256 pixels, random $224\times224$ pixel crops were taken, images were horizontally flipped with a probability of 0.5 in order to increase the speaker's viewpoint invariance and each RGB channel was normalized using values expected by the pretrained ResNet50 module.

For the baseline experiments, the training pairs of images were sampled at random. In contrast, for the experiment involving similar training pairs the images were sampled such that the distractor depicted objects similar to the target. This was controlled via the category annotations accompanying each image. The training pairs were constructed so that they were annotated with at least three same basic-level categories, or, in case there were less than three, if all annotated categories matched.

\subsection{3Dshapes}
\label{ds:3dshapes}
The 3Dshapes dataset introduced by \textcite{burgess20183d} contains 480.000 synthetically generated images. These images depict different three-dimensional geometric objects in abstract space, systematically varying along six different features. These features are the color of the ground on which the object resides (ten values), the color of the background walls (ten values), the color of the object itself (ten values), the type of object (four values), its size (eight values) and its position within the depicted room (15 values). %\pt{Table X depicts / describes the possible values of each dimension}. 
The $64\times64$ pixel images are labeled with six-dimensional vectors, each position representing the value of one of the six features. 
\pt{Show some examples}.
The dataset was downloaded from \url{https://storage.cloud.google.com/3d-shapes/3dshapes.h5}.
%\pt{XXX images were used for training the agents in the 3Dshapes experiment, and YYY images were used for testing.}

\subsubsection{Caption Generation}

The dataset provides symbolic feature labels of the images. However, since the goal of presented work is training artificial agents to use natural language in communication, English captions were generated for this dataset. More specifically, two datasets were generated. 

The first dataset is tailored towards studying the effect of including maximally exhaustive image descriptions in the training data. Therefore, generated captions include natural language descriptions of all six features and their values, appearing in different syntactic constructions. \pt{Figure XY shows an example image along with sample generated captions.} The syntactic variations were introduced in order to avoid potential effects of order in which descriptions of certain features might occur.
The captions were generated by constructing generation rules and sampling all possible resulting sentences using the \texttt{nltk} package \parencite{bird2006nltk}. The resulting dataset contains \pt{check: 48} synonymous captions per image, five of which are sampled at random for training in order to match the MS COCO conditions. The captions were build using a vocabulary of 49 tokens. Maximal caption length is 25 tokens, and mean caption length is \pt{TBD}. 
%\pt{The most prominent grammatical structure in the dataset was...}

The second dataset provides the comparison for investigating the role of having exhaustive captions in the training dataset, while keeping constant the vocabulary size. That is, this dataset contains \textit{non-}exhaustive captions which only mention three out of six features of each image. These captions were generated using the same procedure. \pt{X} captions per image were generated. %The maximal length in this dataset is ...; the average length is ... .
\pt{TODO: again, examples will be added}
%\pt{potential confound with length.}
The comparability of the vocabulary sizes is important in that it determines the difficulty of learning the policy with REINFORCE which represents the functional learning in these experiments \parencite[cf.][]{havrylov2017emergence}.

The generation rules and the corresponding code can be accessed under \url{https://github.com/polina-tsvilodub/3dshapes-language}.

%\pt{TBD. Cite Bruni's paper with the grammar stuff.}

Akin to MS COCO experiments, baseline experiments are conducted on random target-distractor pairs. For the similar pairs experiment, the distractor is chosen so that it matches the target with respect to at least three out of six features. For instance, if the target depicts a tiny red cube in the left corner on yellow floor and with blue background, the distractor could depict a tiny blue cube in the right corner on yellow floor and with green background.

%This dataset was chosen due to the systematicity of its content , allowing to generate natural language captions for each image which would exhaustively describe all features of the images. Since no such captions exist to the author's knowldege, they are generated as part of this thesis. To this end, a base context-free grammar (CFG) containing production rules for captions was manually created. The syntactic rules were constructed on the basis of examples of captions created by the author for samples from the dataset. Each caption must contain descriptions of all six feature dimensions. The terminal production rules mapping the pre-terminal to natural language labels are chosen for each image based on its unique feature value configuration. For each image, \pt{X captions are sampled from all the possible productions allowed by the grammar}.

%\pt{Discuss aspects to be taken care of, like vocab size, caption lengths etc}.

\section{Architecture}
The architecture of the agents follows \textcite{lazaridou2020multi}, except minor details to be explained below. As described in Chapter \ref{chapter03}, \textcite{lazaridou2020multi} explore different ways to parametrize the speaker agent, and the current work replicates the ``multi-task learning'' parametrization (p. 5). This choice is motivated by the fact that among their architectures, this is the only one where the speaker agent learns to produce messages and its core image captioning capability while having access to both the target and the distractors, as opposed to models relying on sampling captions from a pretrained single-image captioning models. The chosen architecture is used in all experiments.

Both agents have two components: a visual embedding module which takes as input the target and distractor images, and a language module. More specifically, the visual module for both agents embeds image features which were extracted using the same pre-trained ResNet50 model \parencite{he2016deep}. The weights were accessed through the \texttt{torchvision.models} API \parencite{marcel2010torchvision}. Features from all training images of the dataset were extracted and saved and then retrieved when training all agents. 
The language module differs for the two agents, so details are provided below. 
The code for all experiments can be found under \url{https://github.com/polina-tsvilodub/mSc-thesis}. 

\subsection{Speaker}
The speaker receives as input tuples \texttt{(targets, distractors)}, where \texttt{targets} as well as \texttt{distractors} are features extracted from the ResNet50 for the sampled pairs, respectively. That is, the speaker knows which of the the two images in a given iteration of the reference game is the target for which the message should be produced. The speaker then produces a probability distribution over vocabulary tokens of the message given the input $P(m \mid i_t, i_d)$, parametrized by the speaker model parameters $\theta_S$. From a reinforcement learning perspective, training the speaker amounts to estimating the parametrized speaker policy $\pi_{\theta_s}(m \mid i_t, i_d)$.

The speaker model consists of a linear layer which projects the 2048-dimensional image features to 512-dimensional embedding space. The linear layer is first applied to both the target images and then to the distractors, resulting in embeddings $i_t$ and $i_d$, respectively. The vectors are concatenated to $[i_t; i_d]$, the target embeddings always being the first ones, such that the speaker network implicitly knows which features represent the target. These 1024-dimensional vectors are used as additional context in the speaker's language module. The core of the module is the recurrent Long Short-Term Memory (LSTM) cell \parencite{hochreiter1997long}. More specifically, the language module consists of three layers: an embedding layer, mapping the vocabulary to 512-dimensional word embeddings, a one-layer LSTM with 512-dimensional hidden and cell states; and a linear layer on top, mapping the last hidden state of the LSTM to a distribution over the vocabulary. The size of the vocabulary depends on the dataset (see above).

During the reference game training, the speaker receives pairs of images, embeds them and prepends the concatenated embedding to the special \texttt{START} token. This 1536-dimensional embedding is then passed through the LSTM and the linear output layer. The next token is sampled from a categorical distribution parametrized with the probabilities computed from the hidden state in the previous timestep. The sampled token is concatenated with the image embeddings and input to the next timestep. This procedure repeats until the \texttt{END} token is sampled or the maximum caption length is achieved. The \textit{baseline} experiments use the pure decoding strategy; additional experiments investigate the effects of using greedy decoding (see Chapter \ref{chapter02}). This architecture is also a common practice in image captioning literature (cf. Chapter \ref{chapter02}). Another possible approach that could be explored in future work would be to initialize the hidden states of the LSTM with the visual embeddings. In this case, the hidden and cell states of the LSTM were initialized with random values sampled from the standard normal distribution.The embedding and last linear layers' weights were initialized with random values sampled from the uniform distribution between -0.1 and 0.1, the biases were initialized with zeroes.  
This architecure results in a total of \pt{10,429,910 check} trainable parameters.  %\pt{Make a graphic of the models and / or the reference game}.

\subsection{Listener}
The listener receives as input the tuples \texttt{(images1, images2, messages)}, the first two inputs being ResNet50 features of the image pairs, input in random order, such that the agent doesn't know which image is the target. The \texttt{message} is the caption produced by the speaker. The listener then produces scores $P(i|m)$ over images identifying which one is the target, parametrized by the listener model parameters $\theta_L$.

The message is passed to the language module which consists of an embedding layer, mapping the vocabulary to 512-dimensional word embeddings, and a one-layer LSTM with 512-dimensional hidden and cell states. All weights are initialized analagously to the respective weights of the speaker model. The hidden cell state $h_i$ at the last time step of embedding the received message is used as the final message representation. 
The visual module of the listener consists of a linear layer which also projects the 2048-dimensional image features to a 512-dimensional embedding space. The images are passed through the linear layer one after the other, resulting in embeddings $i_1$ and $i_2$. Finally, the listener computes the dot products between each image embedding and the message embedding. The target is sampled from a categorical distribution parametrized with probabilities computed based on the dot products.  
The architecture results in \pt{check 1,049,088} trainable parameters for the listener's visual module and \pt{check 4,176,896} trainable parameters for its LSTM.

%\pt{listener policies.}

\subsection{General Training Details}

All experiments were trained with a batch size of 64 pairs, using the Adam optimizer with a learning rate  of 0.001 \parencite{kingma2014adam}. In the reference game setting, due to computational constraints, the models were trained for two epochs per experiment. \footnote{All experiments were conducted on a MacBook Pro with four Intel cores i7-1068NG7 with CPU @ 2.30GHz. On average, pretraining took 2h/epoch, reference games took 6h/epoch.} 

In the reference games, the speaker parameters $\theta_S$ were updated by optimizing a compound loss function $J_s$. More precisely, the speaker loss consisted of a weighted sum of a functional loss $L_f$ and a structural loss $L_s$. The former was computed via the REINFORCE rule based on the listener's task performance (i.e., referential success). The speaker received the reward 1 if the listener guessed the target successfully, and -1 otherwise. The latter was computed via cross-entropy between the message produced by the speaker and the ground truth caption for the given target image. $L_f$ additionally included an entropy regularization term, weighted by 0.1. The resulting loss is $J_s = (1-\lambda_s)L_f + \lambda_s L_s$. 
The listener parameters $\theta_L$ were updated via cross-entropy loss $J_l$ optimized based on the guessed target classification of the images and the ground truth targets using cross-entropy loss. 


\section{Experiment Procedures and Results}

%\pt{when discussing results and esp PPL, \cite{havrylov2017emergence} also compute encoder PPLs}

This section provides an overview of conducted experiments and discusses their procedures as well as results.

\subsection{Speaker Pretraining}

\begin{figure}
	\centering
	\includegraphics[width=\linewidth]{images/coco_pretraining_losses_ppls.png}
	\caption{Left: Training losses during pretraining the speaker on MS COCO; the colors indicate the pretraining mode, i.e., whether the ground truth caption (teacher-forcing loss, red) or the self-generated token (auto-regressive loss, blue) from the previous timestep was supplied at each timestep. Right: Training perplexity during pretraining of the speaker on MS COCO.}
	\label{fig:coco_pretraining}
\end{figure}  

\begin{figure}
	\centering
	\includegraphics[width=\linewidth]{images/3dshapes_pretraining_losses_ppls.png}
	\caption{Left: Training losses during pretraining the speaker on 3DShapes; the colors indicate the pretraining mode, i.e., whether the ground truth caption (teacher-forcing loss, blue) or the self-generated token (auto-regressive loss, orange) from the previous timestep was supplied at each timestep. Right: Training perplexity during pretraining of the speaker on MS COCO.}
	\label{fig:3dshapes_pretraining}
\end{figure}  

%Table with image caption quality metrics of the pretrained speakers. Add same metrics of the speaker agents after ref games as rows to the same table. 

\begin{table}[]
	\begin{tabularx}{\textwidth}{|X|l|l|l|l|l|l|l|}
		\hline
		\textbf{Model name}                                    & \textbf{B-1} & \textbf{B-2} & \textbf{B-3} & \textbf{B-4} & \textbf{M} & \textbf{C} & \textbf{Val. loss} \\ \hline
		Pretrained MS speaker                             & 0.408           & 0.137           & 0.042           & 0.013           & 0.116           & 0.165          & 4.079                    \\ \hline
		MS Baseline, random, $L_s = 0$      &                 &                 &                 &                 &                 &                &                          \\ \hline
		MS Baseline, random, $L_s = 0.5$   &                 &                 &                 &                 &                 &                &                          \\ \hline
		MS Baseline, random, $L_s = 0.75$   & 0.407           & 0.133           & 0.037           & 0.010           & 0.116           & 0.178          & 4.056                    \\ \hline
		MS Baseline, similar, $L_s = 0.75$  &                 &                 &                 &                 &                 &                &                          \\ \hline
		Pretrained 3D speaker                            & 0.671           & 0.343           & 0.163           & 0.077           & 0.283           & 1.026          & 2.091                    \\ \hline
		3D Baseline, random, $L_s = 0$     &                 &                 &                 &                 &                 &                &                          \\ \hline
		3D Baseline, random, $L_s = 0.5$   &                 &                 &                 &                 &                 &                &                          \\ \hline
		3D Baseline, random, $L_s = 0.75$  & 0.672           & 0.340           & 0.159           & 0.074           & 0.284           & 1.068          & 2.089                    \\ \hline
		3D Baseline, similar, $L_s = 0.75$ & 0.673           & 0.335           & 0.152           & 0.070           & 0.279           & 1.033          & 2.103                    \\ \hline
		\pt{FILL ME with more expts} &                 &                 &                 &                 &                 &                &                          \\ \hline
	\end{tabularx}
\caption{\label{tab:eval_metrics_refgame} Caption evaluation metrics and the validation loss on a heldout dataset, computed for the initial pretrained speakers and the speakers after training on the reference game. \textbf{B} denotes BLEU, \textbf{M} the METEOR score, \textbf{C} the CIDEr score, MS the MS COCO dataset and 3D the 3DShapes dataset. ``Baseline'' refers to the setup wherein the listener is trained jointly with the speaker, using pure decoding. ``Random'' refers to speakers trained on random target-distractor pairs; ``similar'' refers to speakers trained on similar target-distractor pairs.}
\end{table}

The speaker agents were pretrained for both the MS COCO and 3Dshapes experiments. Conceptually, the model was pretrained in order to learn the statistical properties of English, i.e., ``learn to speak'', before being finetuned on the functional reference game task. This is can be motivated as providing general \textit{task-uncoditional} linguistic capabilities to the speaker which she can the transfer to specific tasks.

Both speakers were pretrained in a supervised fashion by sampling pairs tuples of the form $(i_t, i_d, c_t)$ where $i_t, i_d$ are the target and distractor images, and $c_t$ in the target ground truth caption. The models were pretrained on 30,000 images selected at random from the respective dataset; for each image, the model saw five available ground truth captions. 
The pretraining was accomplished with the cross-entropy loss. Both models were trained for ten epochs.

As described in Section \ref{model_pretraining}, there are different strategies for pretraining the speaker. Based on initial exploratory experiments reported in Appendex \ref{app:grid_search}, all speakers were pretrained using an exponentially decreasing teacher-forcing rate $0.5^{epoch-1}$, transitioning to auto-regressive training with pure decoding. That is, in the auto-regressive mode the model was fed its own predicted token from the previous timestep during training; that token was generated by pure sampling. Speakers for both datasets were pretrained in this set up. Figure \ref{fig:coco_pretraining} shows the pretraining dynamics of the MS COCO speaker, Figure \ref{fig:3dshapes_pretraining} shows the dynamics of the 3Dshapes speaker.

The evaluation of their image captioning capabilities on standard metrics (introduced in Chapter \ref{chapter02}) after pretraining can be found in Table \ref{tab:eval_metrics_refgame}. The metrics were computed in a held out validation split with 5000 unique images and respective captions. Comparing the pretrained speakers to state-of-the-art image captioning model peformance in Table \ref{tab_coco_metrics_ref}, it is apparant that the speakers perform somewhat worse iin terms of standard caption evaluation metrics. It is hypothesized that this is due to the mixed teacher-forcing and auto-regression training mode. It is conjectured that the speaker quality only might affect the absolute performance and drift metric values, but the qualitative comparisons between experiments will remain unaffected because all experiments are based on the same pretrained speakers.

Furthermore, all language drift metrics are computed for the pretrained models which arguably present a baseline for linguistic capabilities, before any deterioration might take place due to task optimization (see Table \ref{tab:drift_metrics_basic}).

\subsection{MS COCO: Baseline Experiments}
\label{expt:coco_baseline}
\pt{TODO: References to precise research question numbers from last chapter will be added.}

\begin{table}[]
	\begin{tabularx}{\textwidth}{|X|l|l|X|X|X|X|}
		\hline
		\textbf{Model name}                                    & \textbf{log $P(m)$} & \textbf{log $P(m \mid i)$} & \textbf{Overlap (d)} & \textbf{Overlap (c)} & \textbf{Listener acc (random)} & \textbf{Listener acc (similar)} \\ \hline
		Pretrained MS speaker               &      -131.598            &           -62.385             &          1.194            &           0.003           & 0.912 (random listeners)                 &                                           \\ \hline
		MS Baseline, random, $L_s = 0$      &                   &                        &                      &                      &                                          &                                           \\ \hline
		MS Baseline, random, $L_s = 0.5$    &                   &                        &                      &                      &                                          &                                           \\ \hline
		MS Baseline, random, $L_s = 0.75$   &       -135.910            &             -71.819          &        1.197              &        0.000              & 0.953                                    &                                           \\ \hline
		MS Baseline, similar, $L_s = 0.75$  &                   &                        &                      &                      &                                          &                                           \\ \hline
		Pretrained 3D speaker                            &       -195.753            &         -145.638               &        5.428              &      0.001                & 0.953 (random listeners)                 & 0.808 (similar listeners)                 \\ \hline
		3D Baseline, random, $L_s = 0$     &                   &                        &                      &                      &                                          &                                           \\ \hline
		3D Baseline, random, $L_s = 0.5$   &                   &                        &                      &                      &                                          &                                           \\ \hline
		3D Baseline, random, $L_s = 0.75$  &       -195.495        &           -147.313           &          5.247            &         0.001             & 0.979                                    &                        0.959                   \\ \hline
		3D Baseline, similar, $L_s = 0.75$\footnote{The speaker is evaluated on random pairs here} &      -198.189             &       -140.786                 &           5.578           &        0.001              & 0.878                      &            0.906                        \\ \hline
		&                   &                        &                      &                      &                                          &                                           \\ \hline
	\end{tabularx}
\caption{\label{tab:drift_metrics_basic} Language drift metrics and listener test accuracies on different pairs. 
	``Baseline'' refers to the setup wherein the listener is trained jointly with the speaker, using pure decoding. MS refers to the MS COCO dataset, 3D ro the 3DShapes one. ``Random'' refers to speakers trained on random target-distractor pairs; ``similar'' refers to speakers trained on similar target-distractor pairs. ``Overlap (d)'' refers to the discrete overlap metric, ``overlap (c)'' to continuous overlap.}
\end{table}

The baseline experiment on MS COCO was conducted on 30,000 images which were not used during pretraining. The target-distractor pairs were constructed at random, and each target appeared in five pairs such that the structural loss could be minimized with respect to all five available ground truth captions. Pure decoding was used for decoding the speaker's message, and the weight of the structural loss was $L_s = 0.75$. The weight of the functional loss was $L_f = 0.25$, respecitvely. These configurations are treated as the baseline experiment since preliminary explorations revealed that these are the minimal requirements for a successful reference game (see Appendix \ref{app:grid_search} for details). The training dynamics can be seen in Figure \ref{fig:coco_baseline_075_speaker_loss} (overall speaker loss) and Figure \ref{fig:coco_baseline_075_listener_acc} (listener accuracy).

\begin{figure}
	\centering
	\includegraphics[width=\linewidth]{images/coco_baseline_random_075_speaker_loss.png}
	\caption{Total speaker train loss in the baseline MS COCO experiment (pure decoding, $L_s = 0.75$). \pt{The style of the plot will be adjusted to the previous ones.}}
	\label{fig:coco_baseline_075_speaker_loss}
\end{figure}

\begin{figure}
	\centering
	\includegraphics[width=\linewidth]{images/coco_baseline_random_075_listener_acc.png}
	\caption{Listener train accuracy in the baseline MS COCO experiment (pure decoding, $L_s = 0.75$) \pt{The style of the plot will be fixed}}
	\label{fig:coco_baseline_075_listener_acc}
\end{figure}

\begin{figure}
	\centering
	\includegraphics[width=\linewidth]{images/coco_baseline_random_075_structural_drift.png}
	\caption{Structural drift computed every 200 training steps on 320 held out images in the baseline MS COCO experiment (pure decoding, $L_s = 0.75$) \pt{The style of the plot will be fixed. The weird spikes of the ground truth drift are because the plot contains a point for each of the three validation steps. Will be averaged in the final plot}} 
	\label{fig:coco_baseline_075_str_drift}
\end{figure}

For computing the task accuracy, listener accuracy was computed on a held out validation set of 1000 pairs of images which weren't part of wither pretraining or the reference game training. The test accuracy of 0.953 in Table \ref{tab:drift_metrics_basic} shows that the agents successfully learned to play the reference game. Furthermore, as expected, the language underwent slight deterioration, both syntactically and semantically, compared to the pretrained speaker. The dynamics of structural drift can be seen in Figure \ref{fig:coco_baseline_075_str_drift}.\footnote{Due to a coding mistake, no analogous plot for semantic drift could be created.} Due to the small size of the decrease, no trend can be observed visually.
Interestingly, a slight increase in the discrete overlap metric can be observed, indicating that the speaker learned to produce messages that are more appropriate ifor the target compared to the distractor, and, therefore, might be more discriminative. On the other hand, the similarity of embeddings of the ground truth caption and the message decreased relative to the distractor similarity, compared to the pretrained speaker. This might be due to the difficulty to propagate the learning signal all the way to the embedding layer.

\pt{Linear regressions will be computed on the drift values collected during the training.}

Additionally, the fine-tuned speaker was also evaluated with standard image captioning metrics. Table \ref{tab:eval_metrics_refgame} shows that caption quality marginally decreased with respect to almost all metrics, confirming the trend shown by the structural and semantic language drift metrics.

\subsection{MS COCO: Similar Pairs Experiments}

% For evaluating the random pairs vs the similar pairs, it would be good to conduct tests with similar and random pairs for both agents, and the especially look at discrete overlaps on the similar pairs, and check if the agents learned to distinguish the images more graunlarly. If not, discuss this as an interesting result, wherein it is actually easier for humans if the discriminative feature is clear as opposed to more distinct images (check if it cognitively plausible though). Find a way to check the granularity of produced descriptions / categories (maybe via simple POS tagging / partial parsing somehow).


\subsection{MS COCO: Fixed Listener Experiments}

\subsection{3Dshapes: Baseline Experiments}
\label{expt:3dshapes_baseline}

\begin{figure}
	\centering
	\includegraphics[width=\linewidth]{images/3dshapes_baseline_random_075_speaker_loss.png}
	\caption{Total speaker train loss in the baseline 3Dshapes experiment (pure decoding, $L_s = 0.75$). \pt{The style of the plot will be adjusted to the previous ones.}}
	\label{fig:3dshapes_baseline_075_speaker_loss}
\end{figure}

\begin{figure}
	\centering
	\includegraphics[width=\linewidth]{images/3dshapes_baseline_random_075_listener_acc.png}
	\caption{Listener train accuracy in the baseline 3Dshapes experiment (pure decoding, $L_s = 0.75$) \pt{The style of the plot will be fixed}}
	\label{fig:3dshapes_baseline_075_listener_acc}
\end{figure}

\begin{figure}
	\centering
	\includegraphics[width=\linewidth]{images/3dshapes_baseline_random_075_str_drift.png}
	\caption{Structural drift computed every 200 training steps on 50 images in the baseline 3Dshapes experiment (pure decoding, $L_s = 0.75$) \pt{The style of the plot will be fixed. The weird spikes of the ground truth drift are because the plot contains a point for each of the three validation steps. Will be averaged in the final plot}} 
	\label{fig:3dshapes_baseline_075_str_drift}
\end{figure}

The configurations as well as the training procedure of this experiment match the configurations of the MS COCO baseline experiment in Section \ref{expt:coco_baseline}. The training dynamics can be seen in Figure \ref{fig:3dshapes_baseline_075_speaker_loss} (total speaker loss) and Figure \ref{fig:3dshapes_baseline_075_listener_acc}. Compared to the reference game on MS COCO, the training of the agents converges faster. This is likely due to the significantly smaller action space (i.e., vocabulary space---49 tokens for 3Dshapes vs. 4052 tokens for MS COCO) that the speaker has to explore. Conversely, it might be the case that the MS COCO baseline agents achieve better results if trained longer.
 
Supporting the visual results, Table \ref{tab:drift_metrics_basic} shows that the agents successfully learned the reference game---the listener test accuracy is 0.979, even outperforming the MS COCO baseline experiment.  
The dynamics of structural drift can be seen in Figure \ref{fig:3dshapes_baseline_075_str_drift}.\footnote{Due to a coding mistake, the language drift and validation loss computed during training of the baseline random pairs experiment on 3Dshapes were computed on 50 images from the training dataset, not the validation dataset.} 
In this experiment, only a slight semantic drift can be seen in Table \ref{tab:drift_metrics_basic}. The structural changes are negligible. The continuous overlap metric decreased, compared to the pretrained speaker. Interestingly, the magnitude of the structural drift values is much higher for 3Dshapes, indicating that the created captions might in general be less likely under the pretrained Transformer XL model used for the computation. Similarly, the higher magnitude of the semantic drift values indicates that the speaker is generally uncertain when generating captions for these images. This could partly be due to the difference in the 3Dshapes data distribution compared to the ImageNet data in which the visual module of the speaker was pretrained.

\pt{Linear regressions will be computed on the drift values collected during the training.}

Additionally, the fine-tuned speaker was also evaluated with standard image captioning metrics. Table \ref{tab:eval_metrics_refgame} shows that caption quality marginally decreased with respect to almost all metrics except for CIDEr and BLEU-1. Noteworthily, in contrast to the language drift metrics, these metrics are significantly higher for the 3Dshapes dataset compared to MS COCO, indicating that the generated captions have a relatively high overlap with ground truth captions. 

\pt{Effects of varying $\lambda_s$ will also be described here.}

\subsection{3Dshapes: Similar Pairs Experiments}
\label{expt:3dsapes_similar}

\begin{figure}
	\centering
	\includegraphics[width=\linewidth]{images/3dshapes_baseline_similar_075_speaker_loss.png}
	\caption{Total speaker train loss in the similar pairs 3Dshapes experiment (pure decoding, $L_s = 0.75$). \pt{The style of the plot will be adjusted to the previous ones.}}
	\label{fig:3dshapes_similar_075_speaker_loss}
\end{figure}

\begin{figure}
	\centering
	\includegraphics[width=\linewidth]{images/3dshapes_baseline_similar_075_listener_acc.png}
	\caption{Listener train accuracy in the similar pairs 3Dshapes experiment (pure decoding, $L_s = 0.75$) \pt{The style of the plot will be fixed}}
	\label{fig:3dshapes_similar_075_listener_acc}
\end{figure}

\begin{figure}
	\centering
	\includegraphics[width=\linewidth]{images/3dshapes_baseline_similar_075_str_drift.png}
	\caption{Structural drift computed every 200 training steps on 50 images in the similar pairs 3Dshapes experiment (pure decoding, $L_s = 0.75$) \pt{The style of the plot will be fixed. The weird spikes of the ground truth drift are because the plot contains a point for each of the three validation steps. Will be averaged in the final plot}} 
	\label{fig:3dshapes_similar_075_str_drift}
\end{figure}

In this experiment, the target-distractor pairs consisted of similar images. Otherwise, the procedure and configurations are the same as in the baseline experiment in Section \ref{expt:3dshapes_baseline}. 
Figure \ref{fig:3dshapes_similar_075_speaker_loss} indicates that it is much harder for the speaker to produce discriminative messages when the target-distractor pairs are similar, compared to random pairs. That is, there are less options for producing good messages, such that the functional reward signal is much weaker than in the former experiment, resulting in almost absent speaker adaptation. Similarly, the listener is much slower to learn the reference game and the grounding of the speaker's messages to the images (Figure \ref{fig:3dshapes_similar_075_listener_acc}). These dynamics confirm that the agents are sensitive to their visual input, and that the task success is closely dependent on the perceptual difficulty to discriminate the target and distractors.

The hypothesis addressed in this experiment is whether the agents were able to flexibly adapt the specificity of their messages, compared to the random pairs baseline experiment (\textbf{H5}). This was approximately investigated via the distribution of part-of-speech tags in a test set of 1000 target-distractor pairs for which messages were generated. The distributions were compared when the 1000 pairs are random versus similar. The results are shown in Figure \ref{fig:3dshapes_pos}. Based on visual inspection, it can be seen that the distribution of the token categories, especially the modifier tokens like color or size adjectives, did not shift significantly. This suggests that the artificial speaker agent is not as flexible in adapting her messages to differences in the input as human speakers (cf. Chapter \ref{chapter03}). However, it is evident from Figure \ref{fig:3dshapes_similar_075_listener_acc} that the agents did not converge after two epochs, so longer training might provide a different picture. \pt{TODO: compare this with findings by \cite{lazaridou2016multi} and \cite{lee2019countering}.}

\begin{figure}
	\centering
	\includegraphics[width=\linewidth]{images/3dshapes_random_vs_similar_POS_counts.png}
	\caption{Left: POS counts in captions produced by a speaker trained on random pairs (blue) and a speaker trained on similar pairs(orange) for 1000 random test image pairs. Right: POS counts in captions produced by a speaker trained on random pairs (blue) and a speaker trained on similar pairs(orange) for 100 similar test image pairs. \pt{The difference in the number of pairs will be fixed}}
	\label{fig:3dshapes_pos}
\end{figure}

In this experiment, structural drift compared to both the pretrained speaker and the baseline experiment can be observed (see Table \ref{tab:drift_metrics_basic}), which is corroborated by a slight visually apparent trend in Figure \ref{fig:3dshapes_similar_075_str_drift}. However, a little increase in the discrete overlap compared to the pretrained speaker indicates that the speaker improved on the task of producing more discriminative messages.

\subsection{3Dshapes: Fixed Listener Experiments}
\pt{This experiment did not result in successful reference games under the architecture suggestion originally by MF. That is, when the listener picked the referent based on the likelihood of the generated message given the target vs distractor under the pretrained speaker, the reference game didn't work. I take from it that 1) the speaker might be not good enough and is too uncertain about its own captions and 2) the learning signal might be too weak. If it okay, I will try to use a different fixed listener architecture like e.g. in \cite{lazaridou2020multi}.}

\subsection{3DShapes: Short Captions Experiment}

\section{Language Drift Hypotheses: Discussion}


\chapter{Discussion}
\label{chapter06}
Discussion goes here. 

Competition vs cooperation of agents. 

My intuition that the state space is too large to observe any influence of REINFORCE is supported by marginal comments in \cite{havrylov2017emergence}; therefore, future work should try a combination of the structural loss with a Gumbel-Softmax component.

My architecture, esp for MS COCO, can essentially be interpreted as just show-casing that you can ground a second (receiver) agent against a pretrained image captioner. It is hard to consider this a multi-agent communication result since the effect or reinforce propagating the functional signal is so marginal. 

Discuss how this multi-agent communication is one approach to building pragmatic language behaviour based on neural models. But there are other options with a more explicit underlying cognitive motivation, as e.g. done by Monroe and Potts, but which faces computational challenges. 

\chapter*{Declaration of Authorship}
I hereby certify that the work presented here is, to the best of my knowledge and belief, original and the result of my own investigations, except as acknowledged, and has not been submitted, either in part or whole, for a degree at this or any other university.

\vspace{2cm}
Signature:~\makebox[3in]{\hrulefill}

\vspace{1cm}
City, date:~\makebox[3in]{\hrulefill} 

\appendix
\chapter{Appendix}	
\label{appendix}
\section{MS COCO}
The MS COCO dataset \parencite{chen2015microsoft} contains 12 superordinate categories which contain 80 basic-level categories. Table \ref{tab:app_coco_categories} presents the mappings of the categories as well as the occurrence counts in the train split of the dataset. 

\begin{table}[]
	\begin{tabularx}{\linewidth}{|X|X|l|}
		\hline
		\textbf{Superordinate category}                                    & \textbf{Basic-level categories} & \textbf{Counts}  \\ \hline
		Person & person & 42,151\\  \hline
		accessory & backpack, umbrella, andbag, tie, suitcase &11,172 \\  \hline
		indoor & book, clovk, vase, scissors, teddy bear, hair drier, toothbrush & 9,308\\  \hline
		appliance & microwave, oven, toaster, sink, refrigerator & 4,494 \\  \hline
		food & banana, apple, sandwich, orange, broccoli, carrot, hot dog, pizza, donut, cake & 9,740 \\  \hline
		kitchen & bottle, wine glass, cup, fork, knife, spoon, bowl & 11,226\\  \hline
		sports & frisbee, skis, snowboard, sports ball, kite, baseball bat, baseball glove, skateboard, surfboard, tennis racket &16,240 \\  \hline
		furniture & chair, couch, potted plant, bed, dining table, toilet & 17077\\  \hline
		outdoor & traffic light, fire hydrant, stop sign, parking meter, bench & 8,455\\  \hline
		animal & bird, cat, dog, horse, sheep, cow, elephant, bear, zebra, giraffe & 16,352\\  \hline
		electronic & TV, laptop, mouse, remote, keyboard, cell phone & 7,507 \\  \hline
		vehicle & bicycle, car, motorcycle, airplane, bus, train, truck, boat & 18,547\\ 
		\hline
	\end{tabularx}
	\caption{\label{tab:app_coco_categories} MS COCO basic-level object categories mapped to their superordinate categories, along with counts of the superordinate categories in the train split.}
\end{table}

\section{REINFORCE Pseudo-algorithm}
\pt{TODO}

\section{Speaker Architecture Grid Search}
\label{app:grid_search}

For conducting the experiments in this thesis, different hyperparameters of the reference game were explored in a grid search prior to completing the main experiments. The following hyperparameters were considered: the structural loss weight $\lambda_s \in \{0, 0.5, 0.75, 1\}$, the decoding strategy (exponential, greedy, pure, top-k) and different pretrained speakers. Additionally, different structural loss calculations were considered (see below for details). 

Due to computational constraints, not the full space of configurations was explored, but rather representative experiments in this space were conducted. The grid searches proceeded by training the speaker and listener agent with given configurations for one epoch on a reference game on 15,000 images. 

The first grid search over decoding strategies and structural loss weight, given a teacher-forcing only pretrained speaker revealed that a minimal structural weight of 0.75 was necessary for a reference game with pure decoding with above chance performance (Fig. \ref{fig:coco_grid_Ls_decoding_TF_only}). For greedy decoding, the reference games were successful for all weight configurations. The only correctly implemented exponential decoding experiments were conducted with the structural weights 0.75 and 1, and did not show qualitative improvements compared to pure decoding, so this decoding strategy was left out of further experiments. Furthermore, these results indicated that pure decoding based sampling made the policy learning more difficult for the speaker compared to greedy decoding, as can be observed from the magnitude and variance of the speaker loss for $\lambda_s = 0$ (which amounts to functional-only learning; Fig. \ref{fig:coco_grid_Ls_decoding_TF_only}, upper left).

\begin{figure}[h]
	\centering
	\includegraphics[width=\linewidth]{images/grid_search_TF_only_Ls_decoding.png}
	\caption{Grid search results over different $\lambda_s$ weights (columns) and decoding strategies during a reference game. Upper row shows total speaker train losses, bottom row shows listener train accuracies. Note the different y-axis scales of the last two plots in the upper row.}
	\label{fig:coco_grid_Ls_decoding_TF_only}
\end{figure}

The second grid search was conducted over different speaker pretraining strategies, testing the different speakers in reference games with pure and greedy decoding, and structural weights of 0 and 0.75. Speakers pretrained with the following approaches were considered: teacher forcing only pretraining (see Section \ref{model_pretraining} for details on the pretraining modes), constant 0.5 rate teacher forcing with pure auto-regressive decoding, adaptive teacher forcing rate ($0.5^{epoch}$, pure auto-regressive decoding), scheduled sampling (\cite{bengio2015scheduled}, with pure or exponential decoding, k = 30 and k = 150 for the inverse sigmoid teacher forcing rate decay) and auto-regression only pretraining with top-k sampling with a temperature of 2 \parencite[following][]{lazaridou2020multi}.
Based on manual inspection, the constant 0.5 teacher-forcing rate speaker was found to be more susceptible to repeating words during generation compared to the adaptive rate speaker, so that it was left out of further experiments. Similarly, the scheduled speaker was found to generate poor captions (as measured by image captioning metrics, see Tab. \ref{coco_grid_searches_speaker_pretrain}), so it was not used in reference game experiments. 
Therefore, the adaptive rate and the top-k speakers were compared in the grid search (Fig. \ref{fig:coco_grid_pretraining_decoding}). 
They successfully played the reference game with the structural loss weight 0.75, greedy decoding configurations slightly outperforming pure decoding for both speakers. The performance of both speakers is comparable to the respective performance of the teacher-forced only speaker (Fig. \ref{fig:coco_grid_Ls_decoding_TF_only}, third column). However, very high validation perplexities in auto-regressive mode were observed for the teacher-forced speaker, but not the adaptive  and top-k speakers (Tab. \ref{coco_grid_searches_speaker_pretrain}, last row). Auto-regressive decoding is equivalent to sampling a caption and is at the core of the reference game, and is a more cognitively plausible representation of a message generation process. Therefore, the adaptive speaker was chosen for the final experiments, as it slightly outperformed the top-k speaker across decoding strategies. Noteworthily, using a speaker pretrained with both teacher forcing and auto-regression comes at a cost of somewhat decreasing the image caption quality (see Tab. \ref{coco_grid_searches_speaker_pretrain}).
Because considering different message options, as opposed to a single maximum probability token, is intuitively more plausible, pure decoding is considered as the main decoding strategy in the final experiments.\footnote{A more precise investigation of plausibly narrowing down the considered space of token from the entire vocabulary is left for future work} Top-k decoding was also explored as the decoding strategy in reference games, but those experiments did not yield above chance listener performance, so the strategy was not further pursued and these results are omitted.

\begin{figure}[h]
	\centering
	\includegraphics[width=\linewidth]{images/grid_search_pretraining_Ls_decoding.png}
	\caption{Grid search results over different pretrained speaker models, used with different decoding and $\lambda_s$ weights in a reference game. Upper row shows total speaker training losses, lower row shows listener train accuracies. Left column shows functional only training, right column shows the baseline compound training setting. ``TF'' stands for teacher forcing, ``Top-K'' stands for a speaker pretrained with top-k sampling based auto-regression only.}
	\label{fig:coco_grid_pretraining_decoding}
\end{figure}

% Please add the following required packages to your document preamble:
% \usepackage[table,xcdraw]{xcolor}
% If you use beamer only pass "xcolor=table" option, i.e. \documentclass[xcolor=table]{beamer}
\begin{table}[]
	\begin{tabularx}{\textwidth}{|l|ll|ll|ll|}
		\hline
		\textbf{Metric} & \multicolumn{2}{l|}{\textbf{TF only}}   & \multicolumn{2}{X|}{\textbf{$0.5^{epoch}$ TF rate\newline(pure decoding)}} & \multicolumn{2}{l|}{\textbf{Scheduled}} \\ \hline
		Decoding& \multicolumn{1}{l|}{greedy}  & pure     & \multicolumn{1}{l|}{greedy}                       & pure                        & \multicolumn{1}{l|}{greedy}   & pure    \\ \hline
		B-1             & \multicolumn{1}{l|}{0.649}   & 0.465    & \multicolumn{1}{l|}{0.453}                        & 0.446                       & \multicolumn{1}{l|}{0.112}    & 0.318   \\ \hline
		B-2             & \multicolumn{1}{l|}{0.469}   & 0.264    & \multicolumn{1}{l|}{0.294}                        & 0.178                       & \multicolumn{1}{l|}{0.055}    & 0.110   \\ \hline
		B-3             & \multicolumn{1}{l|}{0.324}   & 0.146    & \multicolumn{1}{l|}{0.182}                        & 0.063                       & \multicolumn{1}{l|}{0.023}    & 0.031   \\ \hline
		B-4             & \multicolumn{1}{l|}{0.222}   & 0.080    & \multicolumn{1}{l|}{0.111}                        & 0.023                       & \multicolumn{1}{l|}{0.008}    & 0.008   \\ \hline
		M               & \multicolumn{1}{l|}{0.207}   & 0.142    & \multicolumn{1}{l|}{0.155}                        & 0.126                       & \multicolumn{1}{l|}{0.062}    & 0.093   \\ \hline
		C               & \multicolumn{1}{l|}{0.666}   & 0.294    & \multicolumn{1}{l|}{0.407}                        & 0.220                       & \multicolumn{1}{l|}{0.098}    & 0.133   \\ \hline
		PPL  & \multicolumn{1}{l|}{7924.05} & 11054.95 & \multicolumn{1}{l|}{108.06}                       & 84.86                       & \multicolumn{1}{l|}{103.74}   & 110.69  \\ \hline
		
	\end{tabularx}
\caption{\label{coco_grid_searches_speaker_pretrain}Evaluation results of speaker pretraining on MS COCO with different decoding strategies during pretraining, and different testing decoding strategies. Both standard image captioning metrics and the validation caption perplexity (PPL) are reported. ``TF'' stands for teacher forcing. Scheduled sampling with pure decoding and k=150 is reported.}
\end{table}

Finally, different structural loss calculations in the reference game were considered. More specifically, a supervised calculation, i.e., a teacher forced prediction  based loss computation (``ground truth maximization'' approach)  was compared to the computation of the loss based on the auto-regressively generated caption (with pure decoding). These were compared for the teacher-forced only pretraining speaker and the adaptive speaker, with a structural loss weight of 0.75.
It was found that there only was a difference in performance when the auto-regressive generation proceeded with pure decoding, but not with greedy decoding, therefore, only the pure decoding comparison is shown in Figure \ref{fig:coco_grid_Ls_calc}. Despite the slightly better listener performance under the ground truth maximization approach (dashed lines, right plot), the auto-regressive approach was selected for the final experiments due to its plausibility for the reference game setting. That is, it is considered more plausible that the structural constraints are computed based on messages actually generated by the speaker agent as she sees fit given the image, and not on miracuolously available teacher-forced ground truth messages. 

\begin{figure}[h]
	\centering
	\includegraphics[width=\linewidth]{images/grid_search_Ls_calculation.png}
	\caption{Grid search results over different structural loss computation strategies in a reference game. ``TF'' stands for teacher forcing, ``GT maximization'' for supervised ground truth-based structural loss computation. Left: Total speaker training losses. Right: Listener accuracies.}
	\label{fig:coco_grid_Ls_calc}
\end{figure}

Other parameters like mean baseline subtraction in the REINFORCE update computation, different wntropy regularization weights, and adaptive learning rates for the optimizers were also explored in single experiments, but did not yield observable improvements in terms of reference game performance. Therefore, those experiments are not reported, and these configurations are not used in the final experiments.

\printbibliography
%\bibliography{references}
\end{document}
